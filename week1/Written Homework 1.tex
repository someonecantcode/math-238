\documentclass[12pt]{article}

\usepackage{amsmath, amssymb, amsthm, graphicx}
\usepackage{csquotes, tikz, cancel, framed}


% \geometry{margin=1in}
\usepackage[left=0.5in, right=0.5in, top=0.5in, bottom=1in]{geometry}

\setlength{\parskip}{0.8em}
\setlength{\parindent}{0pt}

\newtheoremstyle{solution}
  {15pt}
  {15pt}
  {}
  {}
  {\bfseries}
  {: }
  { }
  {}

\theoremstyle{solution}
\newtheorem*{solution}{SOLUTION}

\usetikzlibrary{calc,shapes}

\newcommand{\tikzmark}[1]{\tikz[overlay,remember picture] \node (#1) {};}

%------------------------
\begin{document}

\begin{center}
    \Large \textbf{Homework 1: Introduction to Differential Equations} \\
    \normalsize
    MATH 238: Differential Equations \\
    \vspace{0.2cm}
    Name:\underline{\hspace{1cm}Brendan Tea\hspace{1cm}} \hfill Date:\underline{\hspace{0.5cm}\today\hspace{0.5cm}}
\end{center}

% \vspace{0.4cm}

\textbf{Instructions:} You have received a latex document. Write your answer in this document. Write your name and date in the space provided. Show all your work and clearly justify your answers. Recompile your work. This assignment is intended to introduce the basic ideas and language of differential equations and allow you to practice writing in Latex.
\vspace{-0.4cm}
\section*{Problems}

\begin{enumerate}

\item \textbf{What is a Differential Equation?}

Which of the following equations are differential equations? Explain your reasoning.
\begin{enumerate}
    \item $y = 3x^2 + 1$
    \item $\displaystyle \frac{dy}{dx} = 5x^4$
    \item $\displaystyle \frac{d^2y}{dx^2} + y = 0$
    \item $x^2 + y^2 = 9$
\end{enumerate}

\begin{solution}
    \boxed{\text{b \& d}} are differential equations because definition of a differential equation. 

    \blockquote[Zill 2024, p. 3]{%
        An equation containing the derivatives of one or more unknown functions (or 
        dependent variables), with respect to one or more independent variables, is 
        said to be a differential equation (DE).%
    }

    Namely, as you can see in A \& D, there are no derivatives in general. Thus, A \& D are not differential equations.
    B is a differential equation because it has a first derivative with an unknown function $y(x)$ that is respect to one of the independent variables $x$. 
    C is a differential equation because it has a second derivative with an unknown function $y(x)$ that is repsect to one of the variables $x$.
\end{solution}

\item \textbf{Order of a Differential Equation}

Determine the order of each differential equation below. Explain.

\begin{enumerate}
    \item $\displaystyle \frac{dy}{dx} = \sin x$
    \item $\displaystyle \frac{d^2y}{dx^2} - 4\frac{dy}{dx} + y = 0$
    \item $\displaystyle \left(\frac{d^3y}{dx^3}\right)^2 + y = 1$
\end{enumerate}

\newpage

\begin{solution}
    We simply take the highest order of the derivative from each equation.
    \tikzmark{bl}
            \begin{enumerate}
                \item Order 1, \quad $\displaystyle \frac{dy}{dx}$
                \item Order 2, \quad $\displaystyle \frac{d^2y}{dx^2}$
                \item Order 3, \quad $\displaystyle \frac{d^3y}{dx^3}$
            \end{enumerate}
    \tikzmark{br}
\end{solution}
\tikz[overlay,remember picture]{\draw[black]
  ($(bl)+(-0.2em,-0.8em)$) rectangle
  ($(br)+(0.1em,0.6em)$);}
\vspace{-1cm}

\item \textbf{Linear vs. Nonlinear}

State whether each differential equation is \textbf{linear or nonlinear}. Justify your response.

\begin{enumerate}
    \item $\displaystyle \frac{dy}{dx} + 2y = e^x$
    \item $\displaystyle \frac{dy}{dx} = y^2$
    \item $\displaystyle x^2\frac{d^2y}{dx^2} + y = 0$
\end{enumerate}

\begin{solution}
    We use the defination of a linear differential equation.

    \blockquote[Zill 2024, p. 5]{%
        An $n$th-order ordinary differential equation $F\left(x, y, y', \ldots, y^{(n)}\right) = 0$ is said to be
        \textbf{linear} if $F$ is linear in $y, y', \ldots, y^{(n)}$. This means that an $n$th-order ODE is linear when 
       $F\left(x, y, y', \ldots, y^{(n)}\right) = 0$ is the following.
        %
    }
    \begin{equation}
        a_{n}(x) \frac{d^{n} y}{dx^{n}} + a_{n-1}(x) \frac{d^{n-1} y}{dx^{n-1}} + \cdots + a_{1}(x) \frac{d y}{dx}  + a_0(x)y - g(x) = 0
    \end{equation}

    \tikzmark{bl}
            \begin{enumerate}
                \item Linear, \quad \ \ \ \  All derivatives don't have any powers and coefficients are dependent to $x$.
                \item Nonlinear, \quad \hspace{-0.2cm} The $0$th derivative has a power not equal to $1$, thus it is a nonlinear DE. 
                \item Linear, \quad \ \ \ \ All derivatives don't have any powers and coefficients are dependent to $x$
            \end{enumerate}
    \tikzmark{br}
\end{solution}
\tikz[overlay,remember picture]{\draw[black]
  ($(bl)+(43.5em,-0.8em)$) rectangle
  ($(br)+(0.2em,0.9em)$);}
\vspace{-1.25cm}

\item \textbf{Checking a Solution}

Verify whether the given function is a solution of the differential equation.

\begin{enumerate}
    \item Check whether $y = x^2$ is a solution of
    \[
    \frac{dy}{dx} = 2x
    \]
    
    \item Check whether $y = e^{-2x}$ is a solution of
    \[
    \frac{dy}{dx} + 2y = 0
    \]
\end{enumerate}

\newpage
\begin{solution} The process is to check the left hand side is equal to the right hand side.
    \begin{figure}[h]
        \centering
        \begin{minipage}[t]{0.48\textwidth}
            \begin{align*}
                        (a) \quad  & y = 2x \\
                                &  \frac{dy}{dx} = \frac{d}{dx} x^2\\
                \Longrightarrow & \frac{dy}{dx} = 2x \\
                \therefore \quad & \boxed{y = x^2 \text{ is a solution of }  \frac{dy}{dx} = 2x . }
            \end{align*}
        \end{minipage}
        \hfill
        \begin{minipage}[t]{0.49\textwidth}
            \begin{align*}
                        (b) \quad  & y = e^{-2x} \\
                                & \frac{dy}{dx} = \frac{d}{dx} e^{-2x}\\
                                & \frac{dy}{dx} = -2e^{-2x} \\\\
                                & \frac{dy}{dx} + 2y = -2e^{-2x} + 2y \\
                                & \frac{dy}{dx} + 2y = -2e^{-2x} + 2\left(e^{-2x}\right) \\
                \Longrightarrow & \frac{dy}{dx} + 2y = 0\\
               \therefore \quad & \boxed{y = e^{-2x} \text{ is a solution of } \frac{dy}{dx} + 2y = 0.}
            \end{align*}
        \end{minipage}
    \end{figure}
\end{solution}

\item \textbf{Verifying solutions}

Verify that the given function is a solution to the given differential equation.

\begin{enumerate}
    \item $\frac{dy}{dx} + 4xy = 8x^3$, $y = 2x^2 - 1 + c_1e^{-2x^2}$ 
    \item $x^2\frac{dy}{dx} + xy = 10\sin x$, $y = \frac{5}{x} + \frac{10}{x} \int_1^x\frac{\sin t}{t}dt$
\end{enumerate}

\begin{solution} The process is finding the derivative values to substitute in where we can use the same process from question 5 of checking the left hand side is equal to the right hand side.

    \begin{align*}
                (a) \quad & y = 2x^2 -1 + c_1e^{-2x^{2}} \\
                        & \frac{dy}{dx} = \frac{d}{dx} \left(2x^2 -1 + c_1e^{-2x^{2}} \right) \\
                        & \frac{dy}{dx} = 4x -4xc_1e^{-2x^2} \\\\
                        & \frac{dy}{dx} + 4xy = 8x^3 \\
                        & \left(4x -4xc_1e^{-2x^2}\right) + 4x\left(2x^2 -1 + c_1e^{-2x^{2}}\right) = 8x^3 \\
                        & \cancel{4x} \cancel{-4xc_1e^{-2x^2}} + 8x^3 \cancel{-4x}  \cancel{+4xc_1e^{-2x^{2}}} = 8x^3 \\
       \Longrightarrow  & 8x^3 = 8x^3 \\
       \therefore \quad & \boxed{y = 2x^2 - 1 + c_1e^{-2x^2} \text{ is a solution to the differential equation } \frac{dy}{dx} + 4xy = 8x^3}
    \end{align*}

    \begin{align*}
                (b) \quad & y = \frac{5}{x} + \frac{10}{x} \int_1^x\frac{\sin t}{t}dt \\
                        & \frac{dy}{dx} = \frac{d}{dx} \left(\frac{5}{x} + \frac{10}{x} \int_1^x\frac{\sin t}{t}dt\right) \\
                        & \frac{dy}{dx} = -\frac{5}{x^2} - \frac{10}{x^2}\int_1^x\frac{\sin t}{t}dt +\frac{10}{x}\frac{\sin x}{x} \\
                        & \frac{dy}{dx} = -\frac{1}{x^2} \left(5+10\int_1^x\frac{\sin t}{t}dt +10\sin x\right) \\\\
                        & x^2\frac{dy}{dx} + xy = 10\sin x\\
                        & \cancel{x^2}\left(-\frac{1}{\cancel{x^2}} \left(5+10\int_1^x\frac{\sin t}{t}dt +10\sin x\right)\right) + \cancel{x}\left(\frac{5}{\cancel{x}} + \frac{10}{\cancel{x}} \int_1^x\frac{\sin t}{t}dt\right) = 10\sin x \\
                        & \left(-5-10\int_1^x\frac{\sin t}{t}dt+10\sin x\right)+\left(5+10\int_1^x\frac{\sin t}{t}dt\right)=10\sin x\\
                        & 10\sin x + \cancel{\left(-5-10\int_1^x\frac{\sin t}{t}dt\right)+\left(5+10\int_1^x\frac{\sin t}{t}dt\right)}=10\sin x \\
        \Longrightarrow & 10\sin x = 10\sin x \\
       \therefore \quad & \boxed{y = \frac{5}{x} + \frac{10}{x} \int_1^x\frac{\sin t}{t}dt \text{ is a solution to the differential equation } x^2\frac{dy}{dx} + xy = 10\sin x}
    \end{align*}
    
\end{solution}


\item \textbf{General vs. Particular Solutions}

\begin{enumerate}
    \item Explain the difference between a \textbf{general solution} and a \textbf{particular solution} of a differential equation.
    
    \item Identify which of the following is a general solution:
    \[
    y = Ce^{3x}, \quad y = 5e^{3x}
    \]
\end{enumerate}

\vspace{1cm}
\begin{solution}  \ \ \

    \begin{enumerate}
        \item \tikzmark{bl} \hspace{0.125cm}
            A \textbf{general solution} of a DE is solution with parameters usually formed when given initial \ values. 
            A \textbf{particular solution} of a DE is solution that is free of parameters such as constants \ or coefficients. \tikzmark{br} \vspace{0.2cm}
        \item \hspace{-0.22cm} \boxed{ \ y=Ce^{3x}} is a general solution.
    \end{enumerate}

\end{solution}
\tikz[overlay,remember picture]{\draw[black]
  ($(bl)+(-0.2em,1em)$) rectangle
  ($(br)+(35em,-0.5em)$);}
\vspace{-1cm}

\newpage
\item \textbf{Initial Value Problem}

Consider the differential equation
\[
\frac{dy}{dx} = 3x^2
\]

\begin{enumerate}
    \item Find the general solution.
    \item Find the particular solution that satisfies $y(0) = 4$.
\end{enumerate}


\begin{solution} \ \ \

     \begin{figure}[h]
        \centering
        \begin{minipage}[t]{0.48\textwidth}
             \begin{align*}
                            (a) \quad & \frac{dy}{dx} = 3x^2 \\
                                      & dy = 3x^2 dx  \\
                                      & \int y = \int 3x^2 dx \\
                     \Longrightarrow  & \boxed{y = x^3 + C} \text{ is the general solution.}\\
            \end{align*}
        \end{minipage}
        \hfill
        \begin{minipage}[t]{0.49\textwidth}
            \begin{align*}
                (b)\quad &y(x) = x^3 + C \\
                 &y(0) = (0)^3 + C = 4\\
                 &y(0) =  C = 4 \\
                \Longrightarrow & \boxed{y = x^3 + 4} \text{ is the particular solution.}\\
            \end{align*}
        \end{minipage}
    \end{figure} 
\end{solution}

\item \textbf{Modeling with Differential Equations}

Suppose that the rate of change of a population $P(t)$ is proportional to the size of the population.

\begin{enumerate}
    \item Write a differential equation that models this situation.
    \item Identify the dependent and independent variables.
\end{enumerate}


\begin{solution} \ \
    \begin{enumerate}
        \item \boxed{\frac{dP}{dt} = kt, k \in \mathbb{R}}. We are told that the rate of change of a population $P(t)$ is proportional to the size of the population which translates to $\frac{dP}{dt} \propto P$. 
        \item \boxed{P \text{ is dependent. } t \text{ is independent.}} 
    \end{enumerate}
\end{solution}

\item \textbf{Slope Fields (Conceptual)}

Without solving, describe what information a slope field provides about solutions to a differential equation.

\begin{solution} \ \ \

    \tikzmark{bl} \hspace{0.125cm}
        A slope field or a direction field provides information about the gradient as a line segment at all points. 
        This allows us to gauge information of the solutions of differential equation such as when it is 
        increasing or decreasing, flow or direction, and the general shape of the solution curves. 
    \tikzmark{br}
\end{solution}

\end{enumerate}
\tikz[overlay,remember picture]{\draw[black]
  ($(bl)+(-0.2em,1em)$) rectangle
  ($(br)+(4.4em,-1em)$);}
\vspace{-1cm}


\vfill
\begin{center}
\textit{End of Assignment}
\end{center}
\end{document}
