\section{Vector Calculus is all about this}

Generalized Stokes Theorem
\begin{equation}
    \int_{\omega}^{} d \omega = \int_{\partial \Omega}^{} \omega
\end{equation}

You are integrating over the boundary over the region.

\begin{equation}
    \begin{aligned}
        & \text{The "thing" you study:} & y = f(x) \\
        & \text{The first "derivatives"(s) of that thing:} & \frac{dy}{dx} = f'(x) \\
        & \text{(The differential), "what the things looks like at the infinitesimal level":} & dy = f'(x) dx
    \end{aligned}
\end{equation}


When you find the tangent line to a 2D curve, you are taking a differential of $\Delta y=m \Delta x$ which at the infinitesimal becomes a differential: $dy = f'(x) dx$.
You can't see the curvature of the earth. You are basically so small that it is near a infinitesimal level so it a appears like a curve.

\begin{equation}
    dz = f_x \ dx + f_y \ dy
\end{equation}

\begin{equation}
    dz = \nabla f \cdot dr, \quad dr = \langle f_x, f_y \rangle 
\end{equation}

\begin{equation}
    \int_{C}^{} dz = \int_{C} \nabla f \cdot d\vec{r}
\end{equation}