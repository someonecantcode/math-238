\documentclass{article}
\usepackage{amsmath, amssymb, amsthm, graphicx, subcaption}
\usepackage{csquotes, tikz, cancel, framed}

\graphicspath{{./Images}}

% \geometry{margin=1in}
\usepackage[left=0.5in, right=0.5in, top=0.5in, bottom=1in]{geometry}

\setlength{\parskip}{0.8em}
\setlength{\parindent}{0pt}

\newtheoremstyle{solution}
  {15pt}
  {15pt}
  {}
  {}
  {\bfseries}
  {: }
  { }
  {}

\theoremstyle{solution}
\newtheorem*{solution}{SOLUTION}

\usetikzlibrary{calc,shapes}

\newcommand{\tikzmark}[1]{\tikz[overlay,remember picture] \node (#1) {};}

% ------------------------------------------------------------
\title{MATH 238 Homework \#2}
\author{Brendan Tea}
\date{\today}

\begin{document}

\maketitle

\section*{Homework for Section 2.1}

\begin{enumerate}
    \item Use computer software to obtain a direction field for the given differential equation. Also sketch an approximate solution curve passing through each of the given points
    $$
    \frac{dy}{dx} = e^x \sin(y)
    $$
    \begin{enumerate}
        \item $y(-1) = 4.5$
        \item $y(-1) = -4.5$
        \item $y(-1) = 1$
        \item $y(-1) = -1$
    \end{enumerate}
    Give the equilibrium solutions as solution functions.

    \begin{solution} \ \

       \begin{figure}[h]
            \centering
            \begin{minipage}[t]{0.48\textwidth}
                \vspace{0pt}
                \begin{align*}
                            (a) \quad &  \frac{dy}{dx} = e^x \sin(y) \\
                                      & e^x \sin(y) = 0  \\
                                      & e^x \neq 0, \ \sin(y) = 0 \\
                     \Longrightarrow  & \boxed{y =\pi n, 0, \ \in \mathbb{Z}} \text{ is the general solution.}\\
                \end{align*}

                Equilibrium equations occur when $\frac{dy}{dx} = 0$ allowing us to the 
                find the $y(x)$. 
            \end{minipage}
            \hfill
            \begin{minipage}[t]{0.48\textwidth}
                \vspace{0pt}
                \centering
                \includegraphics[width=0.75\textwidth, keepaspectratio]{p1.png}
                  \caption{Direction field of $\frac{dy}{dx} = e^{x}\sin(y)$}
            \end{minipage}
        \end{figure}

    \end{solution}

    \item The given figure represents the graph of $f(x)$ when $dy/dx = f(x)$. By hand, sketch a direction field over an appropriate grid and include it as an image.
    \begin{figure}[h!]
        \centering
        \includegraphics[width=0.5\linewidth]{HW2 Graph 1.png}
    \end{figure}

\newpage

    \item The given figure represents the graph of $f(y)$ when $dy/dx = f(y)$. By hand, sketch a direction field over an appropriate grid and include it as an image.
    \begin{figure}[h!]
        \centering
        \includegraphics[width=0.5\linewidth]{HW2 Graph 2.png}
    \end{figure}

    \item Consider the autonomous first-order differential equation $dy/dx = 2y - y^2$ and the initial condition $y(0)= y_0$. By hand, sketch the graph of a typical solution $y(x)$ when $y_0$ has the given valuesand include them as images.

    \begin{enumerate}
        \item $y_0 < 0$
        \item $0< y_0 < 2$
        \item $y_0 > 2$
    \end{enumerate}
\end{enumerate}

\section*{Homework for Section 2.2}

\begin{enumerate}

    \item Solve the initial-value problem
    $$
    \frac{dy}{x^2+e^x} = \frac{ydx}{y^2+1}, y(0)= 1
    $$

    \item Solve the initial-value problem
    $$
    x + 3y^2 \sqrt{x^1+1} \frac{dy}{dx} = 0, y(0) = 1
    $$

    \item A glucose solution is administered intravenously into the bloodstream at a constant rate $r$. As the glucose is added, it is converted into other substances and removed from the bloodstream at a rate that is proportional to the concentration at that time. Thus a model for the concentration $C = C(t)$ of the glucose solution in the bloodstream is
    $$
    \frac{dC}{dt} = r - kC
    $$
    where $k$ is a positive constant.
    \begin{enumerate}
        \item Suppose that the concentration at time $t=0$ is $C_0$. Determine the concentration at any time $t$ by solving the differential equation.
        \item Assuming that $C0 < \frac{r}{k}$, find limit: $\lim t \to \infty C(t)$ and interpret your answer.
    \end{enumerate}
    
\end{enumerate}

\end{document}
