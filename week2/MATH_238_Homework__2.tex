\documentclass{article}
\usepackage{amsmath, amssymb, amsthm, graphicx}
\usepackage{csquotes, tikz, cancel, framed}

\graphicspath{{./Images}}

% \geometry{margin=1in}
\usepackage[left=0.5in, right=0.5in, top=0.5in, bottom=1in]{geometry}

\setlength{\parskip}{0.8em}
\setlength{\parindent}{0pt}

\newtheoremstyle{solution}
  {15pt}
  {15pt}
  {}
  {}
  {\bfseries}
  {: }
  { }
  {}

\theoremstyle{solution}
\newtheorem*{solution}{SOLUTION}

\usetikzlibrary{calc,shapes}

\newcommand{\tikzmark}[1]{\tikz[overlay,remember picture] \node (#1) {};}

% ------------------------------------------------------------
\title{MATH 238 Homework \#2}
\author{Brendan Tea}
\date{\today}

\begin{document}
\vspace{-5pt}
\maketitle

\section*{Homework for Section 2.1}

\begin{enumerate}
    \item Use computer software to obtain a direction field for the given differential equation. Also sketch an approximate solution curve passing through each of the given points
    $$
    \frac{dy}{dx} = e^x \sin(y)
    $$
    \begin{enumerate}
        \item $y(-1) = 4.5$
        \item $y(-1) = -4.5$
        \item $y(-1) = 1$
        \item $y(-1) = -1$
    \end{enumerate}
    Give the equilibrium solutions as solution functions.

    \begin{solution} \ \

       \begin{figure}[h]
            \centering
            \begin{minipage}[t]{0.48\textwidth}
                \vspace{0pt}
                \begin{align*}
                            (a) \quad &  \frac{dy}{dx} = e^x \sin(y) \\
                                      & e^x \sin(y) = 0  \\
                                      & e^x \neq 0, \ \sin(y) = 0 \\
                     \Longrightarrow  & \boxed{y=0,\ y =\pi n, \ n \in \mathbb{Z}} \text{ are the equilibrium solutions.}
                \end{align*}

                Equilibrium equations occur for autonomous DE's $\frac{dy}{dx} = f(y)$ when  
                $f(c) = 0 \iff y(x) = c$. 
            \end{minipage}
            \hfill
            \begin{minipage}[t]{0.48\textwidth}
                \vspace{0pt}
                \centering
                \includegraphics[width=0.75\textwidth, keepaspectratio]{p1.png}
                  \caption{Computer sketch of direction field of $\frac{dy}{dx} = e^{x}\sin(y)$ through given points.}
            \end{minipage}
        \end{figure}

    \end{solution}

    \newpage
    \item The given figure represents the graph of $f(x)$ when $dy/dx = f(x)$. By hand, sketch a direction field over an appropriate grid and include it as an image.
    % \begin{figure}[h!]
    %     \centering
    %     \includegraphics[width=0.5\linewidth]{HW2 Graph 1.png}
    % \end{figure}

    \begin{solution}

        We can graph the direction field by plugging in each point $(x_0, y_0)$ into the DE and graphing the slope 
        with its corresponding point. A trick we can use is that we can shift the slopes vertically to save time.
        
        \begin{figure}[h]
            \centering
            \begin{minipage}[t]{0.48\textwidth}
                \vspace{0pt}
                \includegraphics[width=1\linewidth]{HW2 Graph 1.png}
                \caption{Given graph of $\frac{dy}{dx} = f(x)$}
            \end{minipage}
            \begin{minipage}[t]{0.48\textwidth}
                \vspace{0pt}
                \includegraphics[width=1\linewidth]{p2.png}
                \caption{Hand drawn direction field of the DE.}
            \end{minipage}
        \end{figure}
    \end{solution}

    \item The given figure represents the graph of $f(y)$ when $dy/dx = f(y)$. By hand, sketch a direction field over an appropriate grid and include it as an image.

    \begin{solution}
        We must be careful here to see that it is $f(y)$. This is autonomous equation and we can use its horizontal translation property to save time.


        \begin{figure}[h]
            \centering
            \begin{minipage}[t]{0.48\textwidth}
                \vspace{0pt}
                \includegraphics[width=.8\linewidth]{HW2 Graph 2.png}
                \caption{Given graph of $\frac{dy}{dx} = f(y)$}
            \end{minipage}
            \begin{minipage}[t]{0.48\textwidth}
                \vspace{0pt}
                \includegraphics[width=.8\linewidth]{p3.png}
                \caption{Hand drawn direction field of the DE.}
            \end{minipage}
        \end{figure}
    \end{solution}

    \newpage
    \item Consider the autonomous first-order differential equation $dy/dx = 2y - y^2$ and the initial condition $y(0)= y_0$. By hand, sketch the graph of a typical solution $y(x)$ when $y_0$ has the given valuesand include them as images.

    \begin{enumerate}
        \item $y_0 < 0$
        \item $0< y_0 < 2$
        \item $y_0 > 2$
    \end{enumerate}

    \begin{solution}
        We can simply graph the slope at a each varying range on the x-axis and use the
        Translation Property of Autonomous ODEs that states that $y_{T}(x) = y(x-x_0)$ is also 
        a solution to $\frac{dy}{dx} = f(y), / y(x_0) = y_0$. This means we can simply copy and paste
        the same slope on the entire x directions.
       
        \begin{enumerate}
            \item $y_0 < 0$
            \begin{figure}[h]
                \centering
                \includegraphics[width=0.4\linewidth]{p4a.png}
            \end{figure}
            \item $0< y_0 < 2$
            \begin{figure}[h]
                \centering
                \includegraphics[width=0.4\linewidth]{p4b.png}
            \end{figure}
            \item $y_0 > 2$
            \begin{figure}[h]
                \centering
                \includegraphics[width=0.4\linewidth]{p4c.png}
            \end{figure}
        
        \end{enumerate}
    \end{solution}
\end{enumerate}

\newpage
\section*{Homework for Section 2.2}

\begin{enumerate}

    \item Solve the initial-value problem
    $$
    \frac{dy}{x^2+e^x} = \frac{ydx}{y^2+1},\ y(0)= 1
    $$

    \begin{solution}
        \begin{align*}
                        \frac{dy}{x^2+e^x} &= \frac{ydx}{y^2+1} \\
                        \frac{y^2+1}{y}dy &= \left(x^2+e^x \right)dx \\
                        \int \frac{y^2+1}{y}dy &= \int x^2+e^xdx\\
                        \int y + y^{-1} dy &= \frac{x^3}{3} + e^x + C \\
       \Longrightarrow  \frac{y^2}{2} + \ln|y| &= \frac{x^3}{3} + e^x + C\\\\
                        y(0) &= 1 \\
                        \frac{(1)^2}{2} + \ln|(1)| &= \frac{(0)^3}{3} + e^{(0)} + C \\
                        \frac{1}{2} + 0 &= 0 + 1 + C\\
        \Longrightarrow & \boxed{C = -\frac{1}{2}, \ \frac{y^2}{2} + \ln|y| = \frac{x^3}{3} + e^x -\frac{1}{2}}
        \end{align*}
    \end{solution}

    \item Solve the initial-value problem
    $$
    x + 3y^2 \sqrt{x+1} \frac{dy}{dx} = 0, \ y(0) = 1
    $$  

    \vspace{-1cm}
    \begin{solution}
        \begin{align*}
            x + 3y^2 \sqrt{x+1} \frac{dy}{dx} &= 0 \\
            3y^2 dy &= \frac{-x}{\sqrt{x+1}} dx \\
            \int 3y^2 dy &= \int \frac{-x}{\sqrt{x+1}} dx \\
            y^3 &= - \int \frac{(x+1) - 1}{\sqrt{x+1}} dx \\
            y^3 &= - \int \frac{(x+1)}{\sqrt{x+1}} - \frac{1}{\sqrt{x+1}} dx \\
            y^3 &= - \int \sqrt{x+1} - \frac{1}{\sqrt{x+1}} dx \\
            \Longrightarrow  y^3 &= - \left(\frac{2}{3}(x+1)^{\frac{3}{2}} -2(x+1)^{\frac{1}{2}} \right) + C \\\\
         y(0) &= 1 \\
         (1)^3 &= - \left(\frac{2}{3}((0)+1)^{\frac{3}{2}} -2((0)+1)^{\frac{1}{2}} \right) + C \\
            \Longrightarrow \quad & \boxed{C = -\frac{1}{3}, \ y^3 = - \left(\frac{2}{3}(x+1)^{\frac{3}{2}} -2(x+1)^{\frac{1}{2}} \right)  -\frac{1}{3}} 
        \end{align*}
    \end{solution}

    \item A glucose solution is administered intravenously into the bloodstream at a constant rate $r$. As the glucose is added, it is converted into other substances and removed from the bloodstream at a rate that is proportional to the concentration at that time. Thus a model for the concentration $C = C(t)$ of the glucose solution in the bloodstream is
    $$
    \frac{dC}{dt} = r - kC
    $$
    where $k$ is a positive constant.
    \begin{enumerate}
        \item Suppose that the concentration at time $t=0$ is $C_0$. Determine the concentration at any time $t$ by solving the differential equation.
        \item Assuming that $C_0 < \frac{r}{k}$, find limit: $\displaystyle \lim_{t \to \infty} C(t)$ and interpret your answer.
    \end{enumerate}
    

    \begin{solution} \ \
    \begin{enumerate}
        \item Given that $C(0) = C_0$, we can solve the differential equation by integrating and 
        Determine the concentration at any time $t$.
             
        \begin{figure}[h]
                \centering
                \begin{minipage}[t]{0.48\textwidth} \
                    \vspace{0pt}
                     \begin{align*}
                        \frac{dC}{dt}  &= r - kC \\
                        \frac{dC}{r - kC}  &= dt \\
                        \int \frac{dC}{r - kC}  &= \int dt \\
                        -\frac{1}{k} \ln|r-kC| &= t + C_1 \\
                        |r-kC| &= e^{-k(t + C_1) } \\
                        r-kC &= e^{-k(t) } e^{-kC_1} \\
                        C(t) &= \frac{r -e^{-k(t) } e^{-kC_1}}{k} \\
                        C(t) &= \frac{r}{k} - \frac{e^{-kC_1}}{k} e^{-kt} \\
                     \Longrightarrow   C(t) &= \frac{r}{k} + A e^{-kt} 
                    \end{align*}   
                \end{minipage}  
                \hfill
                \begin{minipage}[t]{0.48\textwidth}
                        \vspace{0pt}
                        \begin{align*}
                            & \text{Solve for A: } C(0) = C_0 \\
                            C(0) &= C_0 = \frac{r}{k} + A e^{-k(0)} \\
                            A &= \left(C_0 - \frac{r}{k} \right) \\
                            \Longrightarrow \quad & \boxed{C(t) = \frac{r}{k} + \left(C_0 - \frac{r}{k}\right) e^{-kt}}
                        \end{align*}   
                \end{minipage}
        \end{figure}



        \item To find $\displaystyle \lim_{t \to \infty} C(t)$, assuming that $C_0 < \frac{r}{k}$
        implies that the constant $A$ is negative: $A < 0 $. We can interpret this as when the
        concentration is less than $\frac{r}{k}$, the concentration will try to increase towards 
        and approach $\frac{r}{k}$. We can see this happen vice versa when the concentration is larger,
        it decreases and stabalizes back to $\frac{r}{k}$.

           
        \begin{figure}[h]
                \centering
                \begin{minipage}[t]{0.48\textwidth} \
                    \vspace{0pt}
                    \begin{align*}
                        \text{Given } C_0 < \frac{r}{k}, \  C(0) &= \frac{r}{k} + \left(C_0 - \frac{r}{k} \right) < \frac{r}{k} \\
                        & \left(C_0 - \frac{r}{k} \right) < 0 \\
                        \Longrightarrow \quad &  A < 0 \\
                        C(t) &= \frac{r}{k} + A e^{-kt} \\
                        \lim_{t \to \infty} C(t) &= \frac{r}{k} + \lim_{t \to \infty} A e^{-kt} \\
                        \lim_{t \to \infty} C(t) &= \frac{r}{k} + 0 \\
                        \Longrightarrow \quad & \boxed{\lim_{t \to \infty} C(t) = \frac{r}{k}}
                    \end{align*}
                \end{minipage}  
                \hfill
                \begin{minipage}[t]{0.48\textwidth}
                        \vspace{0pt}
                        \includegraphics[width=0.65\linewidth]{q3.png}
                        \caption{Direction field showing general behavior}
                \end{minipage}
        \end{figure}

        \end{enumerate}    

\end{solution}
\end{enumerate}



\end{document}
