\documentclass{article}
\usepackage{amsmath, amssymb, amsthm, graphicx}
\usepackage{pgfplots, tikz, cancel, framed, hyperref}

\graphicspath{{./Images}}


% \geometry{margin=1in}
\usepackage[left=.5in, right=.5in, top=0.75in, bottom=1in]{geometry}
\pgfplotsset{compat=1.18}
\setlength{\parskip}{0.8em}
\setlength{\parindent}{0pt}

\newtheoremstyle{solution}
  {15pt}
  {15pt}
  {}
  {}
  {\bfseries}
  {: }
  { }
  {}

\theoremstyle{solution}
\newtheorem*{solution}{SOLUTION}

\theoremstyle{definition}
\newtheorem{definition}{DEFINITION}

\usetikzlibrary{calc,shapes}

\newcommand{\tikzmark}[1]{\tikz[overlay,remember picture] \node (#1) {};}
\pgfmathdeclarefunction{erf}{1}{%
  \pgfmathparse{(2/sqrt(pi))*((#1) - ((#1)^3)/3 + ((#1)^5)/10 - ((#1)^7)/42 + ((#1)^9)/216)}%
}


\pgfmathdeclarefunction{erfi}{1}{%
  \pgfmathparse{(2/sqrt(pi))*((#1) + ((#1)^3)/3 + ((#1)^5)/10 + ((#1)^7)/42 + ((#1)^9)/216)}%
}



\pgfplotsset{my style/.append style={axis x line=middle, axis y line=
middle, xlabel={$x$}, ylabel={$y$} }}

% ------------------------------------------------------------

\title{MATH 238 Homework 6}
\author{Brendan Tea}
\date{\today}

\begin{document}
\maketitle
\tableofcontents

\section*{Homework for Section 4.4: 50 points}
\addcontentsline{toc}{section}{Homework for Section 4.2: 50 points}
\vspace{-0.5cm}
In Problems 27 - 36 solve the given initial-value problem.

\subsection*{Problem 4.4.30}
\addcontentsline{toc}{subsection}{Problem 4.4.30}
\vspace{-0.5cm}

\begin{flalign*}
    & \text{Given: } y''+4y'+4y=\left(3+x\right)e^{-2x}; \quad y(0) = 2, \ y'(0) = 5& \\
    & \text{Find:  } \hspace{0.15cm} y(x) &
\end{flalign*}
\begin{solution}
    When we are solving for our general solution, we must first solve for our
    complementary solution, using auxiliary equations, and solve for
    our paricular solution, using the method of undetermined coefficients.
    Then we can use our IVP to solve for our constants.

Solving auxiliary equation for $y_c(x)$.
    \begin{align*}
        & y''+4y'+4y=0 \\
        & m^{2}+4m+4=0 \\
        & m = (m+2)^{2}, \ m = -2, -2 \\
    \Longrightarrow  & \boxed{y_c(x) = C_{1} e^{-2x} + C_{2} xe^{-2x}}
    \end{align*}    

Method of undetermined coefficients for $y_p(x)$.
Notice how there are multiple repeated roots for $e^{-2x}$.
    \begin{align*}
    \text{Assume: }     & y_p(x) = x^{2}(Ax+B)e^{-2x} \\
                        & y_p(x) = (Ax^{3}+Bx^{2})e^{-2x} \\
                        & y_p'(x) = (3Ax^2 + 2Bx - 2Ax^3 - 2Bx^2)e^{-2x} \\
                        & y_p'(x) = (-2Ax^3 + 3Ax^2 - 2Bx^2 + 2Bx)e^{-2x} \\
                        & y_p''(x) = (-6Ax^2 + 6Ax - 4Bx + 2B +4Ax^3-6Ax^2+4Bx^2-4Bx)e^{-2x} \\
                        & y_p''(x) = (4Ax^3-12Ax^2+4Bx^2 + 6Ax - 8Bx + 2B  )e^{-2x}
    \end{align*}
Substitute $y_p(x)$ and solve for constants.
    \begin{align*}
        & y_p(x)''+4y_p(x)'+4y_p(x) = \left(3+x\right)e^{-2x} \\
        & (4Ax^3-12Ax^2+4Bx^2 + 6Ax - 8Bx + 2B  )\cancel{e^{-2x}} + 4(-2Ax^3 + 3Ax^2 - 2Bx^2 + 2Bx)\cancel{e^{-2x}} + 4(Ax^{3}+Bx^{2})\cancel{e^{-2x}} = \left(3+x\right)\cancel{e^{-2x}} \\
        & (\cancel{4Ax^3}\cancel{-12Ax^2}+\cancel{4Bx^2} + 6Ax \cancel{- 8Bx} + 2B ) + (\cancel{-8Ax^3} + \cancel{12Ax^2} \cancel{-8Bx^2} +\cancel{ 8Bx}) + (\cancel{4Ax^3}+\cancel{4Bx^2}) = 3+x \\
        & 6Ax + 2B = 3 + x\\
    \Longrightarrow    & \begin{cases}
            & 6Ax = x \\
            & 2B = 3 \\
        \end{cases} \\
        & A = \frac{1}{6}, \ B = \frac{3}{2} \\
    \Longrightarrow & \boxed{y_p(x) = \left(\frac{1}{6}x^{3}+\frac{3}{2}x^{2}\right)e^{-2x}}
    \end{align*}

Combine to create $y_g(x)$.
    \begin{align*}
        & y_g(x) = y_c(x) + y_p(x)  \\
    \Longrightarrow    & \boxed{y_g(x) = C_{1} e^{-2x} + C_{2} xe^{-2x} + \left(\frac{1}{6}x^{3}+\frac{3}{2}x^{2}\right)e^{-2x}}
    \end{align*}

Now we have to solve the IVP and obtain our constants.
\begin{align*}
    & y(0) = 2, \ y'(0) = 5 \\
    & y_g(x) = C_{1} e^{-2x} + C_{2} xe^{-2x} + \left(\frac{1}{6}x^{3}+\frac{3}{2}x^{2}\right)e^{-2x} \\
    & y_g'(x) = -2C_{1} e^{-2x} + C_{2}e^{-2x} -2C_{2} xe^{-2x} + \left(\frac{1}{2}x^{2}+3x\right)e^{-2x} -2 \left(\frac{1}{6}x^{3}+\frac{3}{2}x^{2}\right)e^{-2x} \\
    & y_g'(x) = -2C_{1} e^{-2x} + C_{2}e^{-2x} -2C_{2} xe^{-2x} + \left(\frac{1}{2}x^{2}+3x\right)e^{-2x} + \left(-\frac{1}{3}x^{3}-3x^{2}\right)e^{-2x} 
\end{align*}

Plugging in.
\begin{align*}
    & y(0) = 2 \\
    & y_g(0) = C_{1} e^{-2(0)} + C_{2} (0)e^{-(0)} + \left(\frac{1}{6}(0)^{3}+3(0)^{2}\right)e^{-2(0)} = 2 \\
    & y_g(0) = C_{1} = 2 \\
    & y'(0) = 5 \\
    & y_g'(0) = -2C_{1} e^{-2(0)} + C_{2}e^{-2(0)} -2C_{2} (0)e^{-2(0)} + \left(\frac{1}{2}(0)^{2}+3(0)\right)e^{-2(0)} + \left(-\frac{1}{3}(0)^{3}-3(0)^{2}\right)e^{-2(0)}  = 5 \\
    & y_g'(0) = -2C_{1} + C_{2} = 5 \\
    & y_g'(0) = -2(2) + C_{2} = 5 \\
    &y_g'(0) = C_{2} = 9
\end{align*}

Combine to create $y(x)$.

\begin{align*}
    & C_{1} = 2, \ C_{2} = 9 \\
    & y_g(x) = C_{1} e^{-2x} + C_{2} xe^{-2x} + \left(\frac{1}{6}x^{3}+\frac{3}{2}x^{2}\right)e^{-2x} \\
\Longrightarrow & \boxed{y_g(x) = 2e^{-2x} + 9 xe^{-2x} + \left(\frac{1}{6}x^{3}+\frac{3}{2}x^{2}\right)e^{-2x}}
\end{align*}

\end{solution}

\subsection*{Problem 4.4.34}
\addcontentsline{toc}{subsection}{Problem 4.4.34}
\vspace{-0.5cm}
\begin{flalign*}
    & \text{Given: } \frac{d^{2}x}{dt^{2}} + \omega^{2}x = F_{0} \cos (\gamma t); \quad x(0) = 0, \ x'(0) = 0 & \\
    & \text{Find:  } \hspace{0.15cm} x(t) &
\end{flalign*}

\begin{solution}
    When we are solving for our general solution, we must first solve for our
    complementary solution, using auxiliary equations, and solve for
    our paricular solution, using the method of undetermined coefficients.
    Then we can use our IVP to solve for our constants.

Solving auxiliary equation for $x_c(t)$.
    \begin{align*}
        & \frac{d^{2}x}{dt^{2}} + \omega^{2}x = F_{0} \cos \gamma t \\
        & m^{2}+\omega^{2}=0 \\
        & m = \pm i \omega \\
    \Longrightarrow  & \boxed{x_c(t) = C_{1} \cos(\omega t) + C_{2} \sin(\omega t)}
    \end{align*}    

Method of undetermined coefficients for $x_{p}(t)$.
 \begin{align*}
    \text{Assume: }     & x_p(t) = A\cos(\gamma t)  + B\sin (\gamma t) \\
                        & x_p'(t) = -\gamma A\sin (\gamma t) + \gamma B \cos (\gamma t) \\
                        & x_p''(t) = -\gamma^{2} A\cos (\gamma t) - \gamma^{2} B \sin (\gamma t)
    \end{align*}

Substitute $x_{p}(t)$ and solve for constants.
    \begin{align*}
        & \frac{d^{2}x_{p}(t)}{dt^{2}} + \omega^{2}x_{p}(t) = F_{0} \cos (\gamma t) \\
        & \left(-\gamma^{2} A\cos (\gamma t) - \gamma^{2} B \sin (\gamma t)\right) + \omega^{2}\left(A\cos(\gamma t)  + B\sin (\gamma t)\right) = F_{0} \cos (\gamma t) \\
        &  \left(-\gamma^{2} A\cos (\gamma t) - \gamma^{2} B \sin (\gamma t)\right) + \left(\omega^{2}A\cos(\gamma t)  +\omega^{2} B\sin (\gamma t)\right) = F_{0} \cos (\gamma t)\\
        & A(\omega^{2} - \gamma^{2}) \cos(\gamma t) + B(\omega^{2} - \gamma^{2}) \sin(\gamma t) = F_{0} \cos (\gamma t) \\
    \Longrightarrow    & \begin{cases}
            & A(\omega^{2} - \gamma^{2}) \cos(\gamma t)  = F_{0} \cos (\gamma t) \\
            & B(\omega^{2} - \gamma^{2}) \sin(\gamma t) = 0 \\
        \end{cases} \\
        & A = \frac{F_{0}}{\omega^{2} - \gamma^{2}}, \ B = 0 \\
    \Longrightarrow & \boxed{x_p(t) = \frac{F_{0}}{\omega^{2} - \gamma^{2}} \cos(\gamma t) }
    \end{align*}

Combine to create $x_g(t)$.
    \begin{align*}
        & x_g(t) = x_c(t) + x_p(t)  \\
    \Longrightarrow    & \boxed{x_g(t) = C_{1} \cos(\omega t) + C_{2} \sin(\omega t) + \frac{F_{0}}{\omega^{2} - \gamma^{2}} \cos(\gamma t)}
    \end{align*}

Now we have to solve the IVP and obtain our constants.
    \begin{align*}
        & x(0) = 0, \ x'(0) = 0 \\
        & x_g(t) = C_{1} \cos(\omega t) + C_{2} \sin(\omega t) + \frac{F_{0}}{\omega^{2} - \gamma^{2}} \cos(\gamma t) \\
        & x_g'(t) = -\omega C_{1} \sin(\omega t) +\omega C_{2} \cos(\omega t) - \frac{F_{0} \gamma}{\omega^{2} - \gamma^{2}} \sin(\gamma t) 
    \end{align*}

Plugging in.
    \begin{align*}
        & x_g(0) = C_{1} + \frac{F_{0}}{\omega^{2} - \gamma^{2}} = 0 \\
        & x_g(0) = C_{1} = -\frac{F_{0}}{\omega^{2} - \gamma^{2}} \\
        & x_g'(0) = -\omega C_{1} \sin(\omega (0)) +\omega C_{2} \cos(\omega (0)) - \frac{F_{0} \gamma}{\omega^{2} - \gamma^{2}} \sin(\gamma (0)) \\
        & x_g'(0) = \omega C_{2} = 0
        & x_g'(0) = C_{2} = 0 
    \end{align*}

Combine to create $x(t)$.
    \begin{align*}
        & C_{1} = -\frac{F_{0}}{\omega^{2} - \gamma^{2}}, \ C_{2} = 0 \\
        & x_g(t) = C_{1} \cos(\omega t) + C_{2} \sin(\omega t) + \frac{F_{0}}{\omega^{2} - \gamma^{2}} \cos(\gamma t) \\
        & x(t) = -\frac{F_{0}}{\omega^{2} - \gamma^{2}} \cos(\omega t) + \frac{F_{0}}{\omega^{2} - \gamma^{2}} \cos(\gamma t) \\
    \Longrightarrow & \boxed{x(t) = \frac{F_{0}}{\omega^{2} - \gamma^{2}} \left(\cos(\gamma t) - \cos(\omega t) \right)}
    \end{align*}
\end{solution}

\subsection*{Problem 4.4.42}
\vspace{-0.5cm}
\addcontentsline{toc}{subsection}{Problem 4.4.42}
In Problems 41 and \textbf{42} solve the given initial-value problem in which 
the input function $g(x)$ is discontinuous. [\textit{Hint}: Solve each problem on two intervals, and 
then find a solution so that $y$ and $y'$ are continuous at $x \, = \, \pi / 2$ (Problem 41) and
$\boldsymbol{x \, = \, \pi}$ (\textbf{Problem 42})]

\begin{flalign*}
    & \text{Given: } y''-2y'+10y=g(x), \ y(0) = 0, \ y'(0) = 0, \ \text{where} & \\
    & g(x) = \begin{cases}
                  20    & 0 \leq x \leq \pi  \\[2mm]
                  0     & x \geq \pi
              \end{cases} & \\
    & \text{Find:  } \hspace{0.15cm} y(x) &
\end{flalign*}

\begin{solution}
    This problem is unique as it is two nonhomogenous problems in one. We will solve both general solutions normally
    as the past two. We will be able to find our first set of constants for the range of $\leq x \leq \pi$
    by just normally plugging in. For the range of $x \geq \pi$, we will have to make it satisfy continuity and
    smoothness continuity. In other words, $y_1(\pi) = y_2(\pi)$ for continuity and $y_1'(\pi) = y_2'(\pi)$ for
    smoothness.

    Solving for $y_{1c}(x)$ for range of $0 \leq x \leq \pi$. 
    \begin{align*}
        & y'' - 2y' + 10y = 20 \\
        & m^2 - 2m + 10 = 0 \\
        & m = \frac{2\pm \sqrt{4-40}}{2} \\
        & m = \frac{2 \pm 6i}{2} \\
        & m = 1 + 3i, 1 - 3i \\
        \Longrightarrow & \boxed{y_{1c}(x) = e^{x} \left(C_{1}\cos\left(3 x\right) + C_{2}\sin\left(3 x\right)\right)}
    \end{align*}

    Solving for $y_{1p}(x)$ for range of $0 \leq x \leq \pi$. 
    \begin{align*}
        \text{Assume: } & y_{1p}(x) = A \\
                        & y_{1p}^{(n)}(x) = 0, \quad n \in \mathbb{Z}^{+}, \ n \geq 1 \\
                        & \cancel{y_{p}''} - 2\cancel{y_{p}'} + 10y_{p} = 20 \\
                        & 10A = 20 \\
                        & A = 2 \\
                        \Longrightarrow & y_{1p}(x) = 52    \end{align*}

    Solving for $y_{1g}(x)$ and $y_{1g}'(x)$.
    \begin{align*}
        & y_{1g}(x) = y_{1c}(x) + y_{1p}(x) \\
        & y_{1g}(x) = C_{1}\cos\left(2 x\right) + C_{2}\sin\left(2 x\right) + 5 \\
        & y_{1g}'(x) = -2C_{1}\sin\left(2 x\right) + 2 C_{2}\cos\left(2 x\right)
    \end{align*}

    Plugging in for constants.
    \begin{align*}
        & y(0) = 1, \ y'(0) = 2 \\
        & y_{1g}(0) = C_{1}\cos\left(2 (0)\right) + C_{2}\sin\left(2 (0) \right) + 5 = 1 \\
        & y_{1g}(0) = C_{1} + 5 = 1 \\
        & y_{1g}(0) = C_{1} = -4 \\
        & y_{1g}'(0) = -2C_{1}\sin\left(2 (0)\right) + 2 C_{2}\cos\left(2 (0)\right) = 2 \\
        & y_{1g}'(0) = 2 C_{2} = 2 \\
        & y_{1g}'(0) = C_{2} = 1
    \end{align*}

    Combine for $y_{1g}(x)$ for range of $0 \leq x \leq \pi$. 
    \begin{align*}
        & C_{1} = -4, \quad C_{2} = 1 \\
        & y_{1g}(x) = C_{1}\cos\left(2 x\right) + C_{2}\sin\left(2 x\right) + 5 \\
    \Longrightarrow & \boxed{y_{1}(x) = -4\cos\left(2 x\right) + \sin\left(2 x\right) + 5}
    \end{align*}

    Now the same thing for $y_{2}(x)$ for the range of $x \geq \pi$. We should recognize this is just the homogenous, complementary part from $y_{1}(x)$ so 
    we can skip and state the following. There is no particular solution as it is homogenous.
    \begin{align*}
        & y'' + 4y = 0 \\
    \Longrightarrow    & y_{2g}(x) = y_{1c}(x) \\
        & \boxed{y_{2g}(x) = C_{1}\cos\left(2 x\right) + C_{2}\sin\left(2 x\right)}
    \end{align*}    

    Now have to have to satisfy continuity and smoothness continuity for our constants.
    \begin{align*}
        & y_{1}(0) = y_{2}(0) \\
        & y_{1g}(x) = -4\cos\left(2 x\right) + \sin\left(2 x\right) + 5 \\
        & y_{1g}(0) = 1 \\
        & y_{2g}(x) = C_{1}\cos\left(2 x\right) + C_{2}\sin\left(2 x\right) \\
        & y_{2g}(0) = C_{1} \\
        & y_{1}(0) = y_{2}(0) \\
        & C_{1} = 1 \\\\
        & y_{1}'(0) = y_{2}'(0) \\
        & y_{1}'(x) = 8\sin\left(2 x\right) + 2 \cos\left(2 x\right) \\
        & y_{1}'(0) = 2 \\
        & y_{2g}'(x) = -2C_{1}\sin\left(2 x\right) + 2C_{2}\cos\left(2 x\right) \\
        & y_{2g}'(0) = 2C_{2} \\
        & y_{1}'(0) = y_{2}'(0) \\
        & 2 = 2C_{2} \\
        & C_{2} = 1 \\\\
        & y_{2g}(x) = C_{1}\cos\left(2 x\right) + C_{2}\sin\left(2 x\right) \\
        \Longrightarrow & \boxed{y_{2}(x) = \cos\left(2 x\right) + \sin\left(2 x\right)}
    \end{align*}

    Combining everything.

    \begin{equation*}
        \boxed{y(x) = \begin{cases}
                  -4\cos\left(2 x\right) + \sin\left(2 x\right) + 5    & 0 \leq x \leq \pi  \\[2mm]
                  \cos\left(2 x\right) + \sin\left(2 x\right)     & x \geq \pi
              \end{cases}}
    \end{equation*}

\end{solution}

\section*{Homework for Section 4.6: 50 points}
\addcontentsline{toc}{section}{Homework for Section 4.3: 50 points}
In Problems 1 - 20 solve each differential equation by 
variation of parameters.

For second order, we will assume our particular solution comes in this form.
\begin{equation}
    y_{p}(x) = u_{1}y_{1} + u_{2}y_{2}
\end{equation}

After a process, we will arrive with these pairs of equations.
\begin{align*}
    \text{Assume: } & u_{1}'(x) y_{1}(x) + u_{2}'(x) y_{2}(x) = 0 \\
    \Longrightarrow & u_{1}'(x)y_{1}'(x) + u_{2}'(x)y_{2}(x) = f(x)
\end{align*}

By Cramer's Rule.
\begin{equation}
    u_{1}'(x) = \frac{W_{1}}{W}
\end{equation}

\begin{equation}
    u_{2}'(x) = \frac{W_{2}}{W}
\end{equation}

\subsection*{Problem 4.6.4}
\addcontentsline{toc}{subsection}{Problem 4.6.4}
\vspace{-0.5cm}
\begin{flalign*}
    & \text{Given: } y''+y=\sec \theta \tan \theta & \\
    & \text{Find:  } y_g(x) &
\end{flalign*}

\begin{solution}
We simply use auxiliary equations for our complimentary equation and 
Cramer's rule to find our $u_{1}'$ and $u_{2}'$ to find our particular solution $y_p(x)$.

Solving for $y_{c}(x)$.
\begin{align*}
    & y''+y=\sec \theta \tan \theta \\
    & m^2 + 1 = 0 \\
    & m = \pm i \\
    \Longrightarrow & \boxed{y_{c}(x) = C_{1} \cos \left(\theta\right) + C_{2} \sin \left(\theta\right)}
\end{align*}

Solving for determinants $W, \, W_{1}, \, W_{2}$.
\begin{align*}
    & y_{p}(\theta) = u_{1}y_{1} + u_{2}y_{2} \\
    & y_{1}(\theta) =  \cos \left(\theta\right), \ y_{2}(\theta) =  \sin \left(\theta\right), \ f(\theta) = \sec \theta \tan \theta \\
    & W = \begin{vmatrix}
         \phantom{+}\cos \left(\theta\right) & \sin \left(\theta\right) \\
         -\sin \left(\theta\right) & \cos \left(\theta\right) \\
    \end{vmatrix} = 1 \\
    & W_{1} = \begin{vmatrix}
         0 & \sin \left(\theta\right) \\
         \sec \theta \tan \theta & \cos \left(\theta\right)  
    \end{vmatrix} = -\tan^{2} \left(\theta\right) \\
    & W_{2} = \begin{vmatrix}
         \phantom{+}\cos \left(\theta\right) & 0 \\
         -\sin \left(x\right) & \sec \theta \tan \theta
    \end{vmatrix} = \tan \left(\theta\right)
\end{align*}
   
Solving for $u_{1}(\theta)$ and $u_{2}(\theta)$.
\begin{align*}
    & u_{1}(\theta) = \int \frac{W_{1}}{W} d\theta \\
    & u_{1}(\theta) = \int -\tan^{2}\left(\theta\right) d\theta \\
    & u_{1}(\theta) = \int \left( 1 - \sec^{2}\left(\theta\right) \right) d\theta \\
    & u_{1}(\theta) = \theta  - \tan\left(\theta\right) \\\\
    & u_{2}(\theta) = \int \frac{W_{2}}{W} d\theta \\
    & u_{2}(\theta) = \int \tan\left(\theta\right) d\theta \\
    & u_{2}(\theta) = \ln |\sec\left(\theta\right)|
\end{align*}

Solving for $y_p\left(\theta\right)$.
\begin{align*}
    & y_{p}(x) = u_{1}y_{1} + u_{2}y_{2} \\
    & y_{p}(x) = \left(\theta  - \tan\left(\theta\right)\right)\left(\cos \left(\theta\right)\right) + \left(\ln |\sec\left(\theta\right)|\right) \left(\sin \left(\theta\right)\right) \\
    & y_{p}(x) = \theta \cos \left(\theta\right) - \sin \left(\theta\right) + \sin \left(\theta\right) \ln |\sec\left(\theta\right)|
\end{align*}

Now combine and sum $y_{c}$ and $y_{p}$. Additionally, see how there is a repeated solution $\sin \left(\theta\right)$ 
that will get swallowed up by constant in $y_{c}$.

\begin{align*}
    & y_{g}(\theta) = y_{c}(\theta) + y_{p}(\theta) \\
    & y_{g}(\theta) = C_{1} \cos \left(\theta\right) + C_{2} \sin \left(\theta\right) + \theta \cos \left(\theta\right) \cancel{- \sin \left(\theta\right)} + \sin \left(\theta\right) \ln |\sec\left(\theta\right)| \\
    \Longrightarrow & \boxed{y_{g}(\theta) = C_{1} \cos \left(\theta\right) + C_{2} \sin \left(\theta\right) + \theta \cos \left(\theta\right) + \sin \left(\theta\right) \ln |\sec\left(\theta\right)|}
\end{align*}

\end{solution}



\subsection*{Problem 4.6.10}
\addcontentsline{toc}{subsection}{Problem 4.6.10}
\vspace{-0.5cm}
\begin{flalign*}
    & \text{Given: } 4y''-y=e^{\frac{x}{2}} + 3& \\
    & \text{Find:  } y_g(x) &
\end{flalign*}

\begin{solution}
We simply use auxiliary equations for our complimentary equation and 
Cramer's rule to find our $u_{1}'$ and $u_{2}'$ to find our particular solution $y_p(x)$.

Solving for $y_{c}(x)$.
\begin{align*}
    & 4y''-y=e^{\frac{x}{2}} + 3 \\
    & 4m^2 - 1 = 0 \\
    & m = \pm \frac{1}{2} \\
    \Longrightarrow & \boxed{y_{c}(x) = C_{1}e^{\frac{x}{2}}+ C_{2} e^{-\frac{x}{2}} }
\end{align*}

Solving for determinants $W, \, W_{1}, \, W_{2}$.
\begin{align*}
    & y_{p}(x) = u_{1}y_{1} + u_{2}y_{2} \\
    & y_{1}(x) =  e^{\frac{x}{2}}, \ y_{2}(x) =  e^{-\frac{x}{2}}, \ f(x) = \frac{e^{\frac{x}{2}} + 3}{4} \\
    & W = \begin{vmatrix}
        e^{\frac{x}{2}} & e^{-\frac{x}{2}} \\
        \frac{1}{2}e^{\frac{x}{2}} &  -\frac{1}{2}e^{-\frac{x}{2}} \\
    \end{vmatrix} \\
    & W =  e^{\frac{x}{2}} e^{-\frac{x}{2}} \begin{vmatrix}
        1 & 1 \\
        \frac{1}{2} & -\frac{1}{2}
    \end{vmatrix} = -1 \\
    & W_{1} = \begin{vmatrix}
        0 & e^{-\frac{x}{2}} \\
        \frac{e^{\frac{x}{2}} + 3}{4} &  -\frac{1}{2}e^{-\frac{x}{2}} \\
    \end{vmatrix}  \\
    & W_{1} = e^{-\frac{x}{2}} 
    \begin{vmatrix}
    0 & 1 \\
     \frac{e^{\frac{x}{2}} + 3}{4} &  -\frac{1}{2} 
    \end{vmatrix} = -\frac{1 + 3e^{-\frac{x}{2}}}{4}\\
    & W_{2} = \begin{vmatrix}
        e^{\frac{x}{2}} & 0 \\
        \frac{1}{2}e^{\frac{x}{2}} &  \frac{e^{\frac{x}{2}} + 3}{4} \\
    \end{vmatrix} \\
    & W_{2} = e^{\frac{x}{2}} 
    \begin{vmatrix}
        1 & 0 \\
        \frac{1}{2} & \frac{e^{\frac{x}{2}} + 3}{4}
    \end{vmatrix} \\
    & W_{2} =  e^{\frac{1}{2}x} \left(\frac{1 + 3 e^{\frac{x}{2}}}{4}\right) \\
    & W_{2} = \frac{e^{x}+3e^{\frac{x}{2}}}{4}
\end{align*}

Solving for $u_{1}(x)$ and $u_{2}(x)$.
\begin{align*}
    & u_{1}(x) = \int \frac{W_{1}}{W} dx \\
    & u_{1}(x) = \int \frac{3e^{-\frac{x}{2}} + 1}{4} dx \\
    & u_{1}(x) = \frac{x -6e^{-\frac{x}{2}} }{4}\\
    & u_{2}(x) = \int \frac{W_{2}}{W} dx \\
    & u_{2}(x) = \int \frac{-e^{x}-3e^{\frac{x}{2}}}{4} dx \\
    & u_{2}(x) = \frac{-e^{x}-6e^{\frac{x}{2}}}{4}
\end{align*}

Solving for $y_p\left(x\right)$.
\begin{align*}
    & y_{p}(x) = u_{1}y_{1} + u_{2}y_{2} \\
    & y_{p}(x) = \left(\frac{x -6e^{-\frac{x}{2}} }{4}\right) \left(e^{\frac{x}{2}}\right) + \left(\frac{-e^{x}-6e^{\frac{x}{2}}}{4}\right) \left(e^{-\frac{x}{2}}\right) \\
    & y_{p}(x) = \frac{xe^{\frac{x}{2}}}{4} - \frac{6}{4} - \frac{e^{\frac{x}{2}}}{4} - \frac{6}{4} \\
    & y_{p}(x) = \frac{xe^{\frac{x}{2}}}{4} - \frac{e^{\frac{x}{2}}}{4} -3
\end{align*}

Now combine and sum $y_{c}$ and $y_{p}$. Additionally, see how there is a repeated solution $- \frac{e^{\frac{x}{2}}}{4}$ 
that will get swallowed up by constant in $y_{c}$.
\begin{align*}
    & y_{g}(x) = y_{c}(x) + y_{p}(x) \\
    & y_{g}(x) = C_{1}e^{\frac{x}{2}}+ C_{2} e^{-\frac{x}{2}} + \frac{xe^{\frac{x}{2}}}{4} \cancel{- \frac{e^{\frac{x}{2}}}{4}} -3 \\
    \Longrightarrow & \boxed{y_{g}(x) = C_{1}e^{\frac{x}{2}}+ C_{2} e^{-\frac{x}{2}} + \frac{xe^{\frac{x}{2}}}{4} -3}
\end{align*}

\end{solution}


\end{document}
