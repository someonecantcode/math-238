\documentclass{article}
\usepackage{amsmath, amssymb, amsthm, graphicx}
\usepackage{pgfplots, tikz, cancel, framed, hyperref}

\graphicspath{{./Images}}


% \geometry{margin=1in}
\usepackage[left=1in, right=1in, top=0.75in, bottom=1in]{geometry}
\pgfplotsset{compat=1.18}
\setlength{\parskip}{0.8em}
\setlength{\parindent}{0pt}

\newtheoremstyle{solution}
  {15pt}
  {15pt}
  {}
  {}
  {\bfseries}
  {: }
  { }
  {}

\theoremstyle{solution}
\newtheorem*{solution}{SOLUTION}

\usetikzlibrary{calc,shapes}

\newcommand{\tikzmark}[1]{\tikz[overlay,remember picture] \node (#1) {};}
\pgfmathdeclarefunction{erf}{1}{%
  \pgfmathparse{(2/sqrt(pi))*((#1) - ((#1)^3)/3 + ((#1)^5)/10 - ((#1)^7)/42 + ((#1)^9)/216)}%
}


\pgfmathdeclarefunction{erfi}{1}{%
  \pgfmathparse{(2/sqrt(pi))*((#1) + ((#1)^3)/3 + ((#1)^5)/10 + ((#1)^7)/42 + ((#1)^9)/216)}%
}



\pgfplotsset{my style/.append style={axis x line=middle, axis y line=
middle, xlabel={$x$}, ylabel={$y$} }}

% ------------------------------------------------------------

\title{MATH 238 Homework 5}
\author{Brendan Tea}
\date{\today}

\begin{document}
\maketitle
\tableofcontents

\section*{Homework for Section 4.2: 50 points}
\addcontentsline{toc}{section}{Homework for Section 4.2: 50 points}
In Problems 1, 2, 3, 4, 5, 6, 7, 8, 9, \textbf{10}, 11, \textbf{12}, 13, \textbf{14}, 15, and 16, 
the indicated function $y_1(x)$ is a solution of the given differential equation. 
Use reduction of order or formula \eqref{eq:reduction}, as instructed, to find a second solution $y_2(x)$.

\begin{equation} \label{eq:reduction}
    y_2(x) = y_1(x) \int \frac{e^{-\int P(x) dx}}{y_{1}^{2}(x)} dx , \quad y''+ P(x)y'+ Q(x)y=0 \tag{5}
\end{equation}

\subsection*{Problem 4.2.010}
\addcontentsline{toc}{subsection}{Problem 4.2.010}
\vspace{-0.5cm}
\begin{flalign*}
    & \text{Given: } x^{2}y''+2xy'-6y=0; \quad y_1 = x^2 & \\
    & \text{Find:  } \hspace{0.15cm} y_2(x) &
\end{flalign*}

\begin{solution}
    We can simply the use the reduction formula \eqref{eq:reduction}. All we need to do is identify
    our $P(x)$ and plug in values, we should arrive at our answer.

    Solving for $P(x)$.
    \begin{align*}
        & x^{2}y''+2xy'-6y=0; \quad y_1 = x^{2} \\
        & y''+\frac{2}{x}y'-\frac{6}{x^{2}}y = 0 \\
        \Longrightarrow & P(x) = \frac{2}{x}
    \end{align*}

    Plugging into \eqref{eq:reduction} and evaluating.
    \begin{align*}
        & y_2(x) = y_1(x) \int \frac{e^{-\int P(x) dx}}{y_{1}^{2}(x)} dx \\
        & y_2(x) = \left(x^2\right) \int \frac{e^{-\int \left(\frac{2}{x}\right) dx}}{\left(x^2\right)^2} dx \\
        & y_2(x) = x^2 \int \frac{e^{-2 \ln|x|}}{x^{4}} dx \\
        & y_2(x) = x^2 \int \frac{\left(\frac{1}{x^{2}}\right)}{x^{4}} dx \\
        & y_2(x) = x^2 \int x^{-6} dx 
    \end{align*}

    Integrating continued.
    \begin{align*}
                       & y_2(x) = x^{2} \left(- \frac{x^{-5}}{5} \right) \\
                       & y_2(x) = - \frac{1}{5x^{3}}  
    \end{align*}
    Because the differential equation is homogeneous and linear, we can drop the constants.
    \begin{equation*}
       \Longrightarrow \boxed{y_2(x) = \frac{1}{x^{3}}, \quad  y_g = c_{1} x^{2} + \frac{c_{2}}{x^{3}}}
    \end{equation*}
\end{solution}

\subsection*{Problem 4.2.012}
\addcontentsline{toc}{subsection}{Problem 4.2.012}
\vspace{-0.5cm}
\begin{flalign*}
    & \text{Given: } 4x^{2}y''+y=0; \quad y_1 = x^{\frac{1}{2}} \ln x & \\
    & \text{Find:  } \hspace{0.15cm} y_2(x) &
\end{flalign*}

\begin{solution}
    We can simply the use the reduction formula \eqref{eq:reduction}. All we need to do is identify
    our $P(x)$ and plug in values, we should arrive at our answer.

    Solving for $P(x)$.
    \begin{align*}
                        & 4x^{2}y''+y=0; \quad y_1 = x^{\frac{1}{2}} \ln x \\ 
                        & 4x^{2}y''+ 0y' + y=0\\
                        & y'' + 0y' + \frac{1}{4x^{2}} y = 0 \\
        \Longrightarrow & P(x) = 0
    \end{align*}

    Plugging into \eqref{eq:reduction} and evaluating.
    \begin{align*}
        & y_2(x) = y_1(x) \int \frac{e^{-\int P(x) dx}}{y_{1}^{2}(x)} dx \\
        & y_2(x) = \left(x^{\frac{1}{2}} \ln x\right) \int \frac{e^{-\int \left(0\right) dx}}{\left(x^{\frac{1}{2}} \ln x\right)^2} dx \\
        & y_2(x) = \left(x^{\frac{1}{2}} \ln x\right) \int \frac{e^{0}}{x (\ln x)^2} dx \\
        & y_2(x) = \left(x^{\frac{1}{2}} \ln x\right) \int \frac{1}{x (\ln x)^2} dx
    \end{align*}
    
    Using u-substitution.
    \begin{align*}
        & u = \ln x \Longrightarrow du = \frac{1}{x} dx \\
        & y_2(x) = \left(x^{\frac{1}{2}} \ln x\right) \int u^{-2} du \\
        & y_2(x) = \left(x^{\frac{1}{2}} \ln x\right) \left(-u^{-1}\right) \\
        & y_2(x) = \left(x^{\frac{1}{2}} \ln x\right) \left(-(\ln x)^{-1}\right) \\
        & y_2(x) = \left(x^{\frac{1}{2}} \cancel{\ln x}\right) \left( \frac{-1}{\cancel{\ln x}}\right) \\
        & y_2(x) = -\sqrt{x}
    \end{align*}

    Because the differential equation is homogeneous and linear, we can drop the constants.
    \begin{equation*}
       \Longrightarrow \boxed{y_2(x) = \sqrt{x}, \quad  y_g = c_1 \sqrt{x} \ln x + c_2 \sqrt{x}}
    \end{equation*}
    
\end{solution}

\subsection*{Problem 4.2.014}
\addcontentsline{toc}{subsection}{Problem 4.2.014}
\vspace{-0.5cm}
\begin{flalign*}
    & \text{Given: } x^{2}y''-3xy'+5y=0; \quad y_1 = x^{2} \cos\left( \ln x \right) & \\
    & \text{Find:  } \hspace{0.15cm} y_2(x) &
\end{flalign*}

\begin{solution}
    We can simply the use the reduction formula \eqref{eq:reduction}. All we need to do is identify
    our $P(x)$ and plug in values, we should arrive at our answer.

    Solving for $P(x)$.
    \begin{align*}
                        & x^{2}y''-3xy'+5y=0; \quad y_1 = x^{2} \cos\left( \ln x \right) \\ 
                        & y''+\frac{-3}{x}y'+\frac{5}{x^2}y=0\\
        \Longrightarrow & P(x) = - \frac{3}{x}
    \end{align*}

    Plugging into \eqref{eq:reduction} and evaluating.
    \begin{align*}
        & y_2(x) = y_1(x) \int \frac{e^{-\int P(x) dx}}{y_{1}^{2}(x)} dx \\
        & y_2(x) = \left(x^{2} \cos\left( \ln x \right)\right) \int \frac{e^{\int \left(\frac{3}{x}\right) dx}}{\left(x^{2} \cos\left( \ln x \right)\right)^2} dx \\
        & y_2(x) = \left(x^{2} \cos\left( \ln x \right)\right) \int \frac{x^3}{x^4 \cos^{2}\left(\ln x\right)} dx \\
        & y_2(x) = \left(x^{2} \cos\left( \ln x \right)\right) \int \frac{1}{x \cos^{2}\left(\ln x\right)} dx 
    \end{align*}

    Using u-substitution.
    \begin{align*}
        & u = \ln x \Longrightarrow du = \frac{1}{x} dx \\
        & y_2(x) = \left(x^{2} \cos\left( \ln x \right)\right) \int \frac{1}{\cos^{2}\left(u\right)} du \\
        & y_2(x) = \left(x^{2} \cos\left( \ln x \right)\right) \int \sec^{2}(u) du \\
        & y_2(x) = \left(x^{2} \cos\left( \ln x \right)\right) \left(\tan(u)\right) \\ 
        & y_2(x) = \left(x^{2} \cos\left( \ln x \right)\right) \left(\tan(\ln x)\right) \\
        &  y_2(x) = x^{2} \sin(\ln x)
    \end{align*}

    Because the differential equation is homogeneous and linear, we can drop the constants.
    \begin{equation*}
       \Longrightarrow \boxed{y_2(x) = x^2\sin \left(\ln \left(x\right)\right), \quad  y_g = c_1 x^{2} \cos\left( \ln x \right)  + c_{2} x^2\sin \left(\ln x\right)}
    \end{equation*}
\end{solution}

\subsection*{Problem 4.2.020}
\addcontentsline{toc}{subsection}{Problem 4.2.020}
In Problems 17, 18, 19, and \textbf{20} the indicated function $y_1(x)$ is a solution of the associated homogeneous equation.
Use the method of reduction \eqref{eq:reduction} of order to find a second solution $y_2(x)$ of the homogeneous equation and 
a particular solution $y_p(x)$ of the given nonhomogeneous equation.
\vspace{-0.5cm}

\begin{flalign*}
    & \text{Given: } y''-4y'+3y=x; \quad y_1 = e^{x} & \\
    & \text{Find:  } \hspace{0.15cm} y_2(x) \text{ and } y_p(x) &
\end{flalign*}

\begin{solution}
    Since this is a nonhomogeneous differential equation, we must use the idea of superposition.
    We will first use the reduction formula \eqref{eq:reduction}, identifying
    our $P(x)$, plugging in values and \textbf{setting the differential equation to zero}, to get our complimentary solution $y_c(x)$ 
    Then, we will use the same process but \textbf{set the differential equation to} {\boldmath{$x$}}. A simpler method
    will be to asssume that the particular solution $y_p(x)$ will hold a linear structure.

    Solving for $P(x)$.
    \begin{align*}
                        & y''-4y'+3y=x; \quad y_1 = e^{x} \\ 
        \Longrightarrow & P(x) = - 4
    \end{align*}

    Plugging into \eqref{eq:reduction} and evaluating.
    \begin{align*}
        & y_2(x) = y_1(x) \int \frac{e^{-\int P(x) dx}}{y_{1}^{2}(x)} dx \\
        & y_2(x) = \left(e^{x}\right) \int \frac{e^{\int \left(4\right) dx}}{\left(e^{x}\right)^2} dx \\
        & y_2(x) = \left(e^{x}\right) \int \frac{e^{4x}}{e^{2x}} dx \\
        & y_2(x) = \left(e^{x}\right) \int e^{2x} dx \\
        & y_2(x) = \left(e^{x}\right) \left(\frac{e^{2x}}{2}\right) \\
        & y_2(x) = \frac{e^{3x}}{2}
    \end{align*}

    Because the differential equation is homogeneous and linear, we can drop the constants.
    \begin{equation*}
    \Longrightarrow \boxed{y_2(x) = e^{3x}}
    \end{equation*}

    To solve for our particular solution $y_p(x)$, we have to make a guess that our particular solution will take form of a linear term.
    % 
    \begin{align*}
        \text{Let } y_p &= A x + B \\
        \text{Then } y_p' &= A \\
        y_p'' &= 0
    \end{align*}

    Substituting.
    \begin{align*}
        & y''-4y'+3y=x \\
        & (0) - 4(A) + 3(Ax+B) = x\\
        & (3A)x + (3B-4A) = x 
    \end{align*}

    Solving for constants.
    \begin{align*}
        & 3A \cancel{x} = \cancel{x} \\
        & 3A = 1 \Longrightarrow A = \frac{1}{3} \\
        & 3B-4A = 0  \\
        & 3B -4\left(\frac{1}{3}\right) = 0 \Longrightarrow B = \frac{4}{9} \\
        \Longrightarrow & \boxed{y_p = \frac{1}{3} x + \frac{4}{9}}
    \end{align*}

    In whole, after combining all components, we arrive here with our general solution.
    \begin{equation*}
        \Longrightarrow \boxed{y_2(x) = e^{3x}, \quad y_p(x) = \left(\frac{x}{3} + \frac{4}{9}\right),\quad y_g = c_1 e^{x} + c_{2} e^{3x} + \left(\frac{x}{3} + \frac{4}{9}\right)}
    \end{equation*}
\end{solution}

\section*{Homework for Section 4.3: 50 points}
\addcontentsline{toc}{section}{Homework for Section 4.3: 50 points}
In Problems 1, 2, 3, 4, 5, 6, 7, 8, 9, 10, 11, 12, and \textbf{14} find the general solution $y_g(x)$ of the given
second-order differential equation.

For auxiliary differential equations represented by $ay''+by'+cy=0$, we know that there exists solutions of
$y = e^{mx}, \quad m \in \mathbb{C}$. We can substitute the derivates and the auxiliary differential equation now 
becomes $e^{mx} \left(am^2+bm+c\right)$ where we can solve for when $m = 0$. With induction, we can apply this to any $n$-order linear 
differential equations.

\begin{equation}
    c_{n} y^{n} + c_{n-1} y^{n-1} + \dots + c_{1} y^{1} + c_{0} y^{0} = 0, \quad y = e^{mx}
\end{equation}

\begin{equation}
    e^{mx} \left( c_{n} y^{n} + c_{n-1} y^{n-1} + \dots + c_{1} y^{1} + c_{0} y^{0}\right) = 0
\end{equation}

\subsection*{Problem 4.3.014}
\addcontentsline{toc}{subsection}{Problem 4.3.014}
\vspace{-0.5cm}
\begin{flalign*}
    & \text{Given: } 2y''-3y'+4y=0& \\
    & \text{Find:  } y_g(x) &
\end{flalign*}

\begin{solution}

    Recognizing the differential equation as an auxiliary equation allows us to solve this very 
    easily and reduce it to a simple quadratic equation.
    \begin{align*}
        & 2y''-3y'+4y=0 \\
        & e^{mx} \left(2m^{2} + -3m + 4\right) = 0, e^{mx} \neq 0 \\
        & 2m^{2} + -3m + 4 = 0 \\
        & m = \frac{3 \pm \sqrt{(-3)^2 - 4(2)(4)}}{2(2)} \\
        & m = \frac{3 \pm \sqrt{-23}}{4} \\
        & m = \frac{3}{4} + i\frac{\sqrt{23}}{4},\quad \frac{3}{4} - i\frac{\sqrt{23}}{4} 
    \end{align*}

    Using distinct conjugate case.
    \begin{align*}
        & m = \alpha \pm i \beta \\
        & y = e^{\alpha x} \left(C_1 \cos(\beta x) + C_2 \sin(\beta x)\right) \\
    \longrightarrow    & \boxed{y = e^\frac{3x}{4} \left(C_1 \cos\left(\frac{\sqrt{23}}{4} x\right) + C_2 \sin\left(\frac{\sqrt{23}}{4} x\right)\right)}
    \end{align*}
    % explain about the case 3 conjugate complex roots
\end{solution}

In Problems 15, 16, 17, 18, 19, 20, 21, 22, 23, \textbf{24}, 25, 26, 27, and 28 
find the general solution of the given higher-order differential equation.

\subsection*{Problem 4.3.024}
\addcontentsline{toc}{subsection}{Problem 4.3.024}
\vspace{-0.5cm}
\begin{flalign*}
    & \text{Given: } y^{(4)} -2y''+y=0& \\
    & \text{Find:  } y_g(x) &
\end{flalign*}

\begin{solution}
    Recognizing the differential equation as an auxiliary equation allows us to solve this very 
    easily and reduce it to a simple quadratic equation.
    \begin{align*}
        & y^{(4)} -2y''+y=0 \\
        & e^{mx} \left(m^{4} -2m^{2} + 1\right) = 0, e^{mx} \neq 0 \\
        & m^{4} -2m^{2} + 1= 0 \\
        & (m^{2}-1)^{2} = 0 \\
        & m = 1, -1, 1, -1 \\
        & y_1(x) = e^{x} + e^{-x}
    \end{align*}

    Using reduction formula \eqref{eq:reduction} to obtain $y_2(x)$.
    \begin{align*}
        & y_2(x) = y_1(x) \int \frac{e^{-\int P(x) dx}}{y_{1}^{2}(x)} dx \\
        & y^{(4)} -2y''+y=0, \quad y_1(x) = e^{m_{1}x}, \ P(x) = 2 \\\\
        & y_2(x) = e^{m_{1} x} \int \frac{e^{2m_{1}x}}{e^{2m_{1}x}} dx = e^{m_{1}} \int dx = xe^{m_{1}x} \\
        & y_2(x) = xe^{x} + xe^{-x} \\
    \end{align*}

    Combining $y_1(x)$ and $y_2(x)$.
    \begin{align*}
                   & y_g(x) = y_1(x) + y_2(x) \\
\Longrightarrow    & \boxed{y_g(x) = C_1 e^{x} + C_2 xe^{x} + C_3 e^{-x} + C_4 xe^{-x}}
    \end{align*}
    % explain about the case 2 repeated real roots
\end{solution}


In Problems 29, 30, 31, \textbf{32}, 33, 34, 35, and 36 solve the given initial-value problem.

\subsection*{Problem 4.3.032}
\addcontentsline{toc}{subsection}{Problem 4.3.032}
\vspace{-0.5cm}
\begin{flalign*}
    & \text{Given: } 4y''+4y'-3y=0, \quad y(0) = 1, \quad y'(0) = 5 & \\
    & \text{Find:  } y(x) &
\end{flalign*}

\begin{solution}
    Recognizing the differential equation as an auxiliary equation allows us to solve this very 
    easily and reduce it to a simple quadratic equation. With the IVP, we will simply need to plug in 
    values and solve for the constants.

    Solving for $y_g(x)$.
    \begin{align*}
                        & 4y''+4y'-3y=0 \\
                        & e^{mx} \left(4m^{2} +4m -3\right) = 0, e^{mx} \neq 0 \\
                        & 4m^{2} + 4m - 3 = 0 \\
                        & m = \frac{-4 \pm \sqrt{(4)^2 - 4(4)(-3)}}{2(4)} \\
                        & m = \frac{-4 \pm \sqrt{16+48}}{8} \\
                        & m = -\frac{1}{2} \pm 1 \\
                        & m = \frac{1}{2}, -\frac{3}{2} \\
     \Longrightarrow    & \boxed{y_g(x) = C_1 e^{\frac{x}{2}} + C_2e^{-\frac{3x}{2}}}
    \end{align*}

    Solving for $y(x)$ and $y'(x)$
    \begin{align*}
        & y(x) = C_1 e^{\frac{x}{2}} + C_2 e^{-\frac{3x}{2}} \\
        & y'(x) = \frac{C_1e^{\frac{x}{2}}}{2} - \frac{3C_2e^{-\frac{3x}{2}}}{2} 
    \end{align*}

    Solving for constants $C_1$ and $C_2$
    \begin{align*}
        & y(0) = 1 \\
        & 1 = C_1 e^{\frac{(0)}{2}} + C_2 e^{-\frac{3(0)}{2}} \\
    \Longrightarrow    & 1 = C_1 + C_2 \\\\
        & y'(0) = 5 \\
        & 5 = \frac{C_1e^{\frac{(0)}{2}}}{2} - \frac{3C_2e^{-\frac{3(0)}{2}}}{2} \\
        & 5 = \frac{C_1}{2} - \frac{3C_2}{2} \\
        \Longrightarrow  &  10 = C_1 - 3C_2
    \end{align*}


    \begin{align*}
        & C_1 + C_2 = 1 \\
        & C_1 - 3C_2 = 10\\
        & \begin{bmatrix}
            1 & 1 \\
            1 & -3
        \end{bmatrix}
        \begin{bmatrix}
            C_1 \\
            C_2
        \end{bmatrix}
        =
        \begin{bmatrix}
            1 \\
            10
        \end{bmatrix} \\
        & \begin{bmatrix}
            C_1 \\
            C_2
        \end{bmatrix}
        =
        \begin{bmatrix}
            1 & 1 \\
            1 & -3
        \end{bmatrix}^{-1}
        \begin{bmatrix}
            1 \\
            10
        \end{bmatrix} \\
        &
        \boxed{
        \begin{bmatrix}
            C_1 \\
            C_2
        \end{bmatrix}
        =
        \frac{1}{4}
        \begin{bmatrix}
            13 \\
            -9
        \end{bmatrix}
        }
    \end{align*}

    Plugging back constants and combining solutions.

    \begin{equation*}
        \Longrightarrow \boxed{y(x) = \frac{13e^{\frac{x}{2}}}{4}  - \frac{9e^{-\frac{3x}{2}}}{4} }
    \end{equation*}
\end{solution}



%old 4.2
%--------- revise
    % \begin{flalign*}
    %    & \text{Assume } y_p = u(x)y_1 \hfill  & 
    % \end{flalign*}
    % \vspace{-0.5cm}
    % \begin{align*}
    %     \text{Let } y_p &= u_{1}y_{1} + u_{2}y_{2} e^x \\
    %     \text{Then } y_p' &= u'e^x + ue^x \\
    %     y_p'' &= u''e^x + 2u'e^x + ue^x
    % \end{align*}

    % Substituting these into the nonhomogeneous equation $y'' - 4y' + 3y = x$:
    % \begin{align*}
    %     (u''e^x + 2u'e^x + ue^x) - 4(u'e^x + ue^x) + 3(ue^x) &= x \\
    %     u''e^x - 2u'e^x &= x \\
    %     u'' - 2u' &= xe^{-x}
    % \end{align*}

    % Let $w = u'$. This gives the first-order linear equation $w' - 2w = xe^{-x}$. \newline
    % Using the integrating factor $\mu(x) = e^{\int -2 dx} = e^{-2x}$:
    % \begin{align*}
    %     e^{-2x} \left(w' - 2w\right) &= e^{-2x} \left(xe^{-x}\right) \\
    %     \frac{d}{dx}[w e^{-2x}] &= xe^{-3x} \\
    %     w e^{-2x} &= \int xe^{-3x} dx \\
    %     w e^{-2x} &= -\frac{1}{3}xe^{-3x} - \frac{1}{9}e^{-3x} \\
    %     u' = w &= -\frac{1}{3}xe^{-x} - \frac{1}{9}e^{-x}
    % \end{align*}

    % Integrating to find $u(x)$:
    % \begin{align*}
    %     \int u' dx &= \int \left( -\frac{1}{3}xe^{-x} - \frac{1}{9}e^{-x} \right) dx \\
    %     u(x) &= \frac{1}{3}xe^{-x} + \frac{4}{9}e^{-x}
    % \end{align*}

    % Multiplying by $y_1 = e^x$ to get the final particular solution:
    % \begin{align*}
    %     y_p(x) &= u(x)y_{1}(x)\\
    %     y_p(x) &= \left( \frac{1}{3}xe^{-x} + \frac{4}{9}e^{-x} \right) e^x \\
    %     \Longrightarrow & \boxed{y_p(x) = \frac{1}{3}x + \frac{4}{9}}
    % \end{align*}
    % %---------

\end{document}
