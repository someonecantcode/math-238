\documentclass{article}
\usepackage{amsmath, amssymb, amsthm, graphicx}
\usepackage{pgfplots, tikz, cancel, framed}

\graphicspath{{./Images}}

% \geometry{margin=1in}
\usepackage[left=0.5in, right=0.5in, top=0.5in, bottom=1in]{geometry}

\setlength{\parskip}{0.8em}
\setlength{\parindent}{0pt}

\newtheoremstyle{solution}
  {15pt}
  {15pt}
  {}
  {}
  {\bfseries}
  {: }
  { }
  {}

\theoremstyle{solution}
\newtheorem*{solution}{SOLUTION}

\usetikzlibrary{calc,shapes}

\newcommand{\tikzmark}[1]{\tikz[overlay,remember picture] \node (#1) {};}
\pgfmathdeclarefunction{erf}{1}{%
  \pgfmathparse{%
    sign(#1) * (1 - exp(-#1*#1*(4/pi + 0.147*#1*#1)/(1 + 0.147*#1*#1)))^(0.5)
  }%
}
\pgfplotsset{my style/.append style={axis x line=middle, axis y line=
middle, xlabel={$x$}, ylabel={$y$} }}

% ------------------------------------------------------------

\title{MATH 238 Homework 3}
\author{Brendan Tea}
\date{\today}

\begin{document}
\maketitle

\section*{Homework for Section 2.3: 50 points}

It is usually easier to do the homework on paper and then type the solutions in the latex document compiling frequently to catch the errors early! Each of the ordinary differential equation in first-order linear and will be referred to as a equation.
\begin{enumerate}
    \item (2 points) On what rectangular regions does the equation below possess a unique solution?
    $$
    a_1(x) \frac{dy}{dx} + a_0(x) y = q(x) 
    $$

    \begin{solution}
        A unique solution exists when both
        $\frac{a_0(x)}{a_1(x)}$ and $\frac{q(x)}{a_1(x)}$ are both
        continuous. In a rectangular region, it would be 
        \boxed{R = [a, b] \times [c, d] \text{ where } a_1(x) \neq 0 \text{ and } a_1(x), \ a_0(x), \ q(x) \text{ are continuous}.}
    \end{solution}

    \item (2 points) On what rectangular regions does the equation below possess a unique solution?
    $$
    \frac{dy}{dx} + p(x) y = q(x) 
    $$

    \begin{solution}
        A unique solution exists in a standard first order linear equation when
        $p(x)$ and $q(x)$ are both continuous. In a rectangular region, it would be
        \boxed{R = [a, b] \times [c, d] \text{ where both } p(x), \ q(x) \text{ are continuous}.}
    \end{solution}

    \newpage
    \item (6 points) Solve the equation below
    $$
    \frac{dP}{dt} + 5tP = P + 2t - 2
    $$

    \begin{solution} \

        To approach this problem, we must first put it into the standard
        first order linear equation $y' + P(x)y = H(x)$. After that, we will find 
        the integrating factor $\mu(x)$.
        \begin{align*}
                            & \frac{dP}{dt} + 5tP = P + 2t - 2 \\
                            & \frac{dP}{dt} + (5t-1)P =  2t - 2 \\
                            & P(t) = 5t-1, \quad H(t) = 2t -2 \\
                            & \mu(t) = e^{\int P(t) dt} \\
                            & \mu(t) = e^{\int 5t-1 dt} \\
                            & \mu(t) = e^{\frac{5t^2}{2} - t} \\
                            & \mu(t) \left(\frac{dP}{dt} + (5t-1)P =  2t - 2\right) \\
            \Longrightarrow & \frac{d}{dt} (\mu(t) P) = \mu(t) (2t - 2) \\
            & \int \frac{d}{dt} (\mu(t) P) dt = \int \mu(t) (2t - 2) dt \\
            & \mu(t) P = \int e^{\frac{5t^2}{2} - t} (2t-2) dt \\
            & e^{\frac{5t^2}{2} - t} P = \int e^{\frac{5}{2}\left(t-\frac{1}{5}\right)^2 - \frac{1}{10}} (2t-2) dt \\
            & e^{\frac{5t^2}{2} - t} P = e^{-\frac{1}{10}} \int e^{\frac{5}{2}\left(t-\frac{1}{5}\right)^2} (2t-2) dt,
            \quad  u = t - \frac{1}{5} \Longrightarrow t = u + \frac{1}{5}, \quad dt = du, \quad (2t-2) = (2u - \frac{8}{5}) \\
            & e^{\frac{5t^2}{2} - t} P = e^{-\frac{1}{10}} \int (2u - \frac{8}{5}) e^{\frac{5}{2}u^2} du  \\
            & e^{\frac{5t^2}{2} - t} P = e^{-\frac{1}{10}} \left(\int 2u e^{\frac{5}{2}u^2} du - \int \frac{8}{5}e^{\frac{5}{2}u^2} du  \right) \\
            & e^{\frac{5t^2}{2} - t} P = e^{-\frac{1}{10}} \left(\frac{2}{5}e^{\frac{5}{2} u^2} - \int \frac{8}{5}e^{\frac{5}{2}u^2} du \right) \\
            & \text{Remark: } \mathrm{erfi}(x) = \frac{2}{\sqrt{\pi}} \int_{0}^{x} e^{t^2} dt  \\
            & e^{\frac{5t^2}{2} - t} P = e^{-\frac{1}{10}} \left(\frac{2}{5}e^{\frac{5}{2} (t - \frac{1}{5})^2} - \frac{8}{5}\sqrt{\frac{\pi}{10}} \mathrm{erfi}\left(\sqrt{\frac{5}{2}} \left(t - \frac{1}{5}\right) \right) + C \right)  \\
            & P(t) = \frac{\left(\frac{2}{5}e^{\frac{5}{2} (t - \frac{1}{5})^2} - \frac{8}{5}\sqrt{\frac{\pi}{10}} \mathrm{erfi}\left(\sqrt{\frac{5}{2}} \left(t - \frac{1}{5}\right) \right) \right) + C} {e^{\frac{5t^2}{2} - t + \frac{1}{10}}} \\
           \Longrightarrow & \boxed{P(t) = \frac{2}{5} - \frac{8}{5} \sqrt{\frac{\pi}{10}} e^{-\frac{5}{2} (t - \frac{1}{5})^2} \mathrm{erfi}\left(\sqrt{\frac{5}{2}} \left(t - \frac{1}{5}\right) \right) + C e^{-\frac{5}{2} (t - \frac{1}{5})^2} }
        \end{align*}
        
    \end{solution}

    \newpage
    \item (7 points) Solve the equation below
    $$
    2L\frac{di}{dt} + 3Ri = E, i(0) = i_0
    $$
    where $L, R$ and $E$ are constants. 

    \begin{solution}
        \begin{align*}
            & 2L\frac{di}{dt} + 3Ri = E, i(0) = i_0 \\
            & \frac{di}{dt} + \frac{3R}{2L}i = E \\
            & \mu(t) = e^{\frac{3R}{2L}t} \\
            & \frac{d}{dt} (e^{\frac{3R}{2L}t} i) = e^{\frac{3R}{2L}t} E \\
            & e^{\frac{3R}{2L}t} i = \int e^{\frac{3R}{2L}t} E dt \\
            & e^{\frac{3R}{2L}t} i = \frac{2L}{3R} e^{\frac{3R}{2L}t} E + C \\
            & i(t) = \frac{2LE}{3R} + C e^{-\frac{3R}{2L}t} \\
            & i(0) = i_0 = \frac{2LE}{3R} + C \\
            & C = i_0 - \frac{2LE}{3R} \\
            \Longrightarrow & \boxed{i(t)  = \frac{2LE}{3R} + \left(i_0 - \frac{2LE}{3R}\right) e^{-\frac{3R}{2L}t}}
        \end{align*}

    \end{solution}

    \item (7 points) Solve the equation below
    $$
    \cos^2x \sin x \frac{dy}{dx} + \left(\cos^3x\right)y= 4
    $$

    \begin{solution}
        \begin{align*}
            & \cos^2x \sin x \frac{dy}{dx} + \left(\cos^3x\right)y= 4 \\
            & \frac{dy}{dx} + \left(\cot x \right)y= 4 \sec^2 x \csc x\\
            & \mu(x) = e^{\ln |\sin x|} = \sin x \\
            & \frac{d}{dx} (e^{\sin x} y) = \sin x 4 \sec^2 x \\
            & \sin x y = \int \sin x 4 \sec^2 x, \quad u = \cos x, \ du = -\sin x  \\
            & \sin x y = -\int 4u^{-2} du \\
            & \sin x y = 4(\cos x)^{-1} + C \\
            \Longrightarrow & \boxed{y(x) = \frac{4}{\sin x \cos x} + C \csc x}
        \end{align*}
    \end{solution}

    \newpage
    \item (10 points) Solve the equation below
    $$
    (x+1) \frac{dy}{dx} + (x+2)y = 2xe^{-x}
    $$

    \begin{solution}
        \begin{align*}
            & (x+1) \frac{dy}{dx} + (x+2)y = 2xe^{-x} \\
            & \frac{dy}{dx} + \frac{x+2}{x+1}y = \frac{2xe^{-x}}{x+1} \\
            & \mu(x) = e^{\int \frac{x+2}{x+1} dx} \\
            & \mu(x) = e^{\int \frac{x+1+1}{x+1} dx} \\
            & \mu(x) = e^{\int 1 + \frac{1}{x+1} dx} \\
            & \mu(x) = e^{x+\ln|x+1|} = (x+1)e^x \\\\
            & \frac{d}{dx} \left((x+1)e^x y \right) = \cancel{(x+1) e^x} \frac{2x\cancel{e^{-x}}}{\cancel{x+1}} \\
            & (x+1)e^x y = \int 2x dx \\
           \Longrightarrow & \boxed{y = \frac{x^2 + C}{(x+1)e^x}}
        \end{align*}
    \end{solution}

    \newpage
    \item (16 points) Solve the equation below
    $$
    \frac{dy}{dx} + 6xy = f(x)
    $$
    where
    $$
    f(x) = 
    \begin{cases} 
    x^2 & x < 1 \\[2mm]
    2x - 1 & x \ge 1
    \end{cases}
    $$
    Graph the right side and one of the the solutions on separate graphs.

    \begin{solution} \ \

        \begin{figure}[h]
                \centering
                \begin{minipage}[t]{0.48\textwidth} \
                    \vspace{0pt}
                        \begin{align*}
                            & \frac{dy}{dx} + 6xy = x^2, \ x < 1 \\
                            & \mu(x) = e^{3x^2} \\
                            & \frac{d}{dx} (e^{3x^2} y) = e^{3x^2} x^2 \\
                            & e^{3x^2} y = \int e^{3x^2} x^2 dx \\
                            & u = x, dv = x e^{3x^2} dx \\
                            & du = dx, v = \frac{1}{6} e^{3x^2}    \\
                            & e^{3x^2} y = \frac{1}{6} xe^{3x^2} - \int \frac{1}{6} e^{3x^2} dx, \ u = \sqrt{3}x, \ \frac{du}{\sqrt{3}} = dx \\
                            & e^{3x^2} y = \frac{1}{6} xe^{3x^2} - \frac{1}{12} \sqrt{\frac{\pi}{3}} \mathrm{erf}(\sqrt{3}x) \\
                            \Longrightarrow & \boxed{y(x) = \frac{x}{6} - \frac{\sqrt{\pi}\mathrm{erf}(\sqrt{3}x)}{12 \sqrt{3}e^{3x^2}} + Ce^{-3x^2}, \quad x < 1}
                        \end{align*}
                \end{minipage}  
                \hfill
                \begin{minipage}[t]{0.48\textwidth}
                        \vspace{0pt}
                            \begin{align*}
                                & \frac{dy}{dx} + 6xy = 2x - 1, \ x \ge 1 \\
                                & \mu(x) = e^{3x^2} \\
                                & \frac{d}{dx} (e^{3x^2} y) = e^{3x^2} (2x - 1) \\
                                & e^{3x^2} y = \int e^{3x^2} (2x - 1) dx \\
                                & e^{3x^2} y = \int 2x e^{3x^2} - e^{3x^2} dx \\
                                & e^{3x^2} y = \frac{1}{3} e^{3x^2} - \frac{1}{2}\sqrt{\frac{\pi}{3}} \mathrm{erf}\left(\sqrt{3}x\right) + C \\
                             \Longrightarrow   & \boxed{y(x) = \frac{1}{3} - \frac{1}{2}\sqrt{\frac{\pi}{3}} \mathrm{erf}\left(\sqrt{3}x\right) e^{-3x^2} + Ce^{-3x^2}, \quad x \ge 1 }
                            \end{align*}
                \end{minipage}
        \end{figure}

        \begin{figure}[h]
            \begin{minipage}[t]{0.48\textwidth}
                \begin{tikzpicture}
                    \begin{axis}[my style, ymin = 0, ymax = 1]
                        \addplot[domain=-10:1,  samples = 100, red]{1/3 - 1/2*sqrt(pi/3)*erf(sqrt(3)*x)*exp(-3*x^2) };
                        \addplot[domain=1:5, samples = 100, blue]{x/6 - (sqrt(pi)*erf(sqrt(3)*x))/(12*sqrt(3)*exp(3*x^2))};
                    \end{axis}
                \end{tikzpicture}
                \caption{Graph of both solution curves when $C = 0$}
            \end{minipage}
            \hfill
            \begin{minipage}[t]{0.48\textwidth}
                \begin{tikzpicture}
                    \begin{axis}[my style, xmin = 0, ymin = 0]
                        \addplot[domain=1:5, samples = 100, blue]{x/6 - (sqrt(pi)*erf(sqrt(3)*x))/(12*sqrt(3)*exp(3*x^2))};
                        
                    \end{axis}
                \end{tikzpicture}
                \caption{Graph of solution curve when $C = 0$}
            \end{minipage}
        \end{figure}





    \end{solution}
\end{enumerate}

\end{document} 