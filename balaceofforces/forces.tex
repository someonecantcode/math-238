\documentclass{article}
\usepackage{amsmath, amssymb, amsthm, graphicx}
\usepackage{pgfplots, tikz, cancel, framed}
\usepackage{biblatex}
\usepackage[colorlinks=false,linkcolor=blue,filecolor=magenta,urlcolor=cyan, pdfpagemode=FullScreen]{hyperref}

\addbibresource{references.bib}
\graphicspath{{./Images}}


% \geometry{margin=1in}
\usepackage[left=1in, right=1in, top=0.75in, bottom=1in]{geometry}
\pgfplotsset{compat=1.18}
\setlength{\parskip}{0.8em}
\setlength{\parindent}{0pt}

\newtheoremstyle{solution}
  {15pt}
  {15pt}
  {}
  {}
  {\bfseries}
  {: }
  { }
  {}

\theoremstyle{solution}
\newtheorem*{solution}{SOLUTION}

\theoremstyle{definition}
\newtheorem{definition}{DEFINITION}

\usetikzlibrary{calc,shapes}

\newcommand{\tikzmark}[1]{\tikz[overlay,remember picture] \node (#1) {};}
\pgfmathdeclarefunction{erf}{1}{%
  \pgfmathparse{(2/sqrt(pi))*((#1) - ((#1)^3)/3 + ((#1)^5)/10 - ((#1)^7)/42 + ((#1)^9)/216)}%
}


\pgfmathdeclarefunction{erfi}{1}{%
  \pgfmathparse{(2/sqrt(pi))*((#1) + ((#1)^3)/3 + ((#1)^5)/10 + ((#1)^7)/42 + ((#1)^9)/216)}%
}



\pgfplotsset{my style/.append style={axis x line=middle, axis y line=
middle, xlabel={$x$}, ylabel={$y$} }}

%------------------------------
\title{Connection of Integration by Parts to Balance of Forces \newline MATH 255 \#25972 Professor Rawlings} 
\author{Brendan Tea}
\date{\today}
%------------------------------
\begin{document}
\maketitle
\tableofcontents

\section{Integration by Parts}
If you're ever randomly asked  in a calculus class, your 
best bet is probably integration by parts.
We're often taught in school to recognize a specific form and
to plug it into a formula. From that point onward, we usually move on 
and never thing much more. 

\begin{definition}
    The formula for Integration by Parts is defined by the following:
    \begin{equation} \label{eq:intbyparts}
        \int u \, dv \ =\ uv - \int v \, du
    \end{equation}
\end{definition}

This is usually derived through the specific form. \cite{enwiki:1321545206}
\begin{align*}
    & \int u(x) v'(x) \, dx = u(x) v(x) - \int u'(x) v(x) \, dx \\
    & \text{Let } u = u(x), \ du = u'(x) \,dx, \ v = v(x),  dv = v'(x) \, dx \\
  \Longrightarrow &  \int u \, dv \ =\ uv - \int v \, du 
\end{align*} 

\subsection{Visual Derivation}

\begin{figure}[h!]
    \centering
    \includegraphics[width=0.5\linewidth]{visual.png}
    \caption{$u$ $v$ visual.}
\end{figure}

Let the area of the rectangular be calculated through width times height and the curve being
in a function of $u$ and $v$. Additionally, let the axes be $u$ and $v$. Then, by simply adding up the 
parts of the whole, we get the area of that same rectangle.

\begin{align*}
    & u(v) \, v(u) = \int u(v) \, dv  + \int v(u) \, du \\
    & \text{Rearranging} \\
    & \int u(v) \, dv \ =\ u(v) \, v(u) - \int v(u) \, du \\ 
    \Longrightarrow & \int u \, dv \ =\ uv - \int v \, du
\end{align*}

\newpage
\subsubsection{Young's Inequality}

\begin{figure}[h!]
    \centering
    \includegraphics[width=0.5\linewidth]{Youngs.png}
    \caption{The area of the rectangle $a$,$b$ can't be larger than sum of the areas under the functions (red) and (yellow). \cite{enwiki:1316559643}}
\end{figure}

As a remark, this is process is very similair to Young's inequality for products which states the following:

\begin{equation}
    a b ~\leq~ \int_0^a f(x)\,dx + \int_0^b f^{-1}(x)\,dx
\end{equation}

\subsection{Differential Derivation}
\label{section:differential}

Another way (simpler) is simply use the product rule in differential form and integrate.
After rearranging, you will arrive at the same solution.

\begin{align*}
    & \text{Given: } u(x), \ v(x) \\
    & \frac{d}{dx} \left( u(x) \, v(x) \right) = u(x) \, \frac{dv}{dx} + v(x) \, \frac{du}{dx} \\
    & d \left( u(x) \, v(x) \right) = u(x) \, dv + v(x) \, du \\
    & \int d\left(u(x) \, v(x)\right) =\ \int u(x) \, dv \ + \int v(x) \, du \\
    & u(x) \, v(x) =\ \int u(x) \, dv \ + \int v(x) \, du \\
\Longrightarrow    & \int u \, dv \ =\ uv - \int v \, du
\end{align*}
\section{Balance of Forces}

The balance of forces is all about equal and opposite forces acting on other
objects. This is an introductory physics topic that is about adding up parts of
forces to a sum. With this idea, we can connect it and show that your simple
physics forces is integration by parts in disguise.

\begin{definition}
    A force is defined by the following.
    \begin{equation}
        \vec{F} = m\vec{a}
    \end{equation}
\end{definition}

\begin{definition}
    When we have a balance of forces, we are at equilibrium with the sum of all total forces
    in every direction.
    \begin{equation}
        \sum \vec{F} = \vec{0}, \, \vec{F}_{\text{net}} = \vec{0}
    \end{equation}
\end{definition}

\subsection{Connection to Integration by Parts}

To arrive back to integration by parts, we will have to do some algebraic manipulation. After which, we will
use our differential derivation product rule trick.

\begin{align*}
     \vec{F}_{\text{net}} &= 0       \\
     \vec{F}_{\text{net}} \, dx &= 0 \, dx \\
     \int \vec{F}_{\text{net}} \, dx &= \int 0 \, dx \\
     \vec{F}_{\text{net}} \, x &= C 
\end{align*}

Using differential form from \ref{section:differential}.
\begin{align*}
    & d(\vec{F}_{\text{net}} x) = \vec{F}_{\text{net}} \, dx + x \, d(\vec{F}_{\text{net}}) + \cancelto{0}{d(C)} \\
    & \int d(\vec{F}_{\text{net}} x) = \int \vec{F}_{\text{net}} \, dx + \int x \, d(\vec{F}_{\text{net}}) \\
    & \vec{F}_{\text{net}} x = \int \vec{F}_{\text{net}} \, dx + \int x \, d(\vec{F}_{\text{net}}) \\
    \Longrightarrow & \boxed{\int \vec{F}_{\text{net}} \, dx = \vec{F}_{\text{net}} x - \int x \, d(\vec{F}_{\text{net}})}
\end{align*}

Just like that, we have the exact same equation. The next question is how should to interpret this.

\subsection{Interpretation}
Right off the bat, we can recognize the form of work. Now, we're left with this weird $\int x \, d(\vec{F}_{\text{net}})$ term. 
Let us start back with this equation instead and work from there.

$$  \vec{F}_{\text{net}} x = \int \vec{F}_{\text{net}} \, dx + \int x \, d(\vec{F}_{\text{net}}) $$

In structurual mechanics, we deal with strain alongside complementary strain energy. 
With the Crotti-Engesser Theorem,


\begin{definition}
    Strain Energy 
    \begin{equation}
        U = \int \vec{F}_{\text{net}} \, dx
    \end{equation}
\end{definition}

\begin{definition}
    Complementary Strain Energy
    \begin{equation}
        U^{*} = \int x \, d\vec{F}_{\text{net}}
    \end{equation}
\end{definition}

We can create a graph that results to the exact same thing we started with.


virtual work and complementary virtual work  
% https://pkel015.connect.amazon.auckland.ac.nz/SolidMechanicsBooks/Part_I/BookSM_Part_I/08_Energy/08_Energy_03_Complementary_Strain_Energy.pdf




\printbibliography[title={References}]
\end{document}