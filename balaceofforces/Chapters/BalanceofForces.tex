\section{Balance of Forces}

The balance of forces is all about equal and opposite forces acting on other
objects. This is an introductory physics topic that is about adding up parts of
forces to a sum. With this idea, we can connect it and show that your simple
physics forces is integration by parts in disguise.

\begin{definition}
    A force is defined by the following.
    \begin{equation}
        \vec{F} = m\vec{a}
    \end{equation}
\end{definition}

\begin{definition}
    When we have a balance of forces, we are at equilibrium with the sum of all total forces
    in every direction.
    \begin{equation}
        \sum \vec{F} = \vec{0}, \, \vec{F}_{\text{net}} = \vec{0}
    \end{equation}
\end{definition}

drafting work work energy theorem
\begin{align*}
     \vec{F}_{\text{net}} &= 0       \\
     \vec{F}_{\text{net}} \, dx &= 0 \, dx \\
     \int \vec{F}_{\text{net}} \, dx &= \int 0 \, dx \\
     \vec{F}_{\text{net}} \, x &= C 
\end{align*}

Using differential form from 1.2.
\begin{align*}
    & d(\vec{F}_{\text{net}} x) = \vec{F}_{\text{net}} \, dx + x \, d(\vec{F}_{\text{net}}) + \cancelto{0}{d(C)}
\end{align*}