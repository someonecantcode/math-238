\section{Balance of Forces}

The balance of forces is all about equal and opposite forces acting on other
objects. This is an introductory physics topic that is about adding up parts of
forces to a sum. With this idea, we can connect it and show that your simple
physics forces is integration by parts in disguise.

\begin{definition}
    A force is defined by the following.
    \begin{equation}
        \vec{F} = m\vec{a}
    \end{equation}
\end{definition}

\begin{definition}
    When we have a balance of forces, we are at equilibrium with the sum of all total forces
    in every direction.
    \begin{equation}
        \sum \vec{F} = \vec{0}, \, \vec{F}_{\text{net}} = \vec{0}
    \end{equation}
\end{definition}

\subsection{Connection to Integration by Parts}

To arrive back to integration by parts, we will have to do some algebraic manipulation. After which, we will
use our differential derivation product rule trick.

\begin{align*}
     \vec{F}_{\text{net}} &= 0       \\
     \vec{F}_{\text{net}} \, dx &= 0 \, dx \\
     \int \vec{F}_{\text{net}} \, dx &= \int 0 \, dx \\
     \vec{F}_{\text{net}} \, x &= C 
\end{align*}

Using differential form from \ref{section:differential}.
\begin{align*}
    & d(\vec{F}_{\text{net}} x) = \vec{F}_{\text{net}} \, dx + x \, d(\vec{F}_{\text{net}}) + \cancelto{0}{d(C)} \\
    & \int d(\vec{F}_{\text{net}} x) = \int \vec{F}_{\text{net}} \, dx + \int x \, d(\vec{F}_{\text{net}}) \\
    & \vec{F}_{\text{net}} x = \int \vec{F}_{\text{net}} \, dx + \int x \, d(\vec{F}_{\text{net}}) \\
    \Longrightarrow & \boxed{\int \vec{F}_{\text{net}} \, dx = \vec{F}_{\text{net}} x - \int x \, d(\vec{F}_{\text{net}})}
\end{align*}

Just like that, we have the exact same equation. The next question is how should to interpret this.

\subsection{Interpretation}
Right off the bat, we can recognize the form of work. Now, we're left with this weird $\int x \, d(\vec{F}_{\text{net}})$ term. 
Let us start back with this equation instead and work from there.

$$  \vec{F}_{\text{net}} x = \int \vec{F}_{\text{net}} \, dx + \int x \, d(\vec{F}_{\text{net}}) $$

In structurual mechanics, we deal with strain alongside complementary strain energy. 
With the Crotti-Engesser Theorem,


\begin{definition}
    Strain Energy 
    \begin{equation}
        U = \int \vec{F}_{\text{net}} \, dx
    \end{equation}
\end{definition}

\begin{definition}
    Complementary Strain Energy
    \begin{equation}
        U^{*} = \int x \, d\vec{F}_{\text{net}}
    \end{equation}
\end{definition}

We can create a graph that results to the exact same thing we started with.


virtual work and complementary virtual work  
% https://pkel015.connect.amazon.auckland.ac.nz/SolidMechanicsBooks/Part_I/BookSM_Part_I/08_Energy/08_Energy_03_Complementary_Strain_Energy.pdf


